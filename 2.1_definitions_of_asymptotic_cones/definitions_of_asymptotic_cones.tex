% !TeX root = definitions_of_asymptotic_cones.tex
\documentclass[a4paper,11pt]{jsarticle}

% 英文化
\usepackage[english]{babel}
% 数式
\usepackage{amsmath}
\usepackage{amsfonts}
\usepackage{amsthm}
\usepackage{amssymb}
\usepackage{bm}
\usepackage{mathtools}
% 画像
\usepackage[dvipdfmx]{graphicx}
% 箇条書き
\usepackage{enumitem}

% 枠付き文章
\usepackage{ascmac}

\newcommand{\QUOTEBOX}[1]{\begin{center}\fbox{\begin{minipage}{.8\textwidth}{#1}\end{minipage}}\end{center}}
\newcommand{\DEFINITION}[2]{\begin{itembox}[l]{\underline{Definition {#1} }}{#2}\end{itembox}}
\newcommand{\PROPOSITION}[2]{\begin{itembox}[l]{\underline{Proposition {#1} }}{#2}\end{itembox}}

\newcommand{\NaturalNumberSet}{\mathbb{N}}
\newcommand{\RealNumberSet}{\mathbb{R}}
\newcommand{\NDemenstionalRealEuclidianSpace}{\mathbb{R}^n}


\SetLabelAlign{Center}{\hfil#1\hfil}
\SetLabelAlign{CenterWithParen}{\hfil(\makebox[1.0em]{#1})\hfil}

\renewcommand{\theenumi}{\roman{enumi}}
\renewcommand{\labelenumi}{(\theenumi)}
\renewcommand{\theenumii}{\theenumi-\alph{enumii}}
\renewcommand{\labelenumii}{(\theenumii)}

\begin{document}

\title{
  2 Asymptotic Cones and Functions \\
  \large 2.1 Definition of Asymptotic Cones}
\author{Ryota Iwamoto}
\date{March 27, 2023}
\maketitle

We use the book; Asymptotic Cones and Functions in Optimization and Variational Inequalities (author: A.AUSLENDER and M.TEBOULLE), pp.25-31.

\QUOTEBOX{
  The set of natural numbers is denoted by $\NaturalNumberSet$, so that $k \in \NaturalNumberSet$ means $k = 1,2, \dots .$ A sequence $\{x_k\}_{k \in \NaturalNumberSet}$ or simply $\{x_k\}$ in $\NDemenstionalRealEuclidianSpace$ is said to converge to $x$ if $\left\lVert x_k - x\right\rVert \rightarrow 0$ as $k \rightarrow \infty$, and this will be indicated by the notation $x_k \rightarrow x$ or $x = \lim_{k \to \infty} x_k$. We say that $x$ is a cluster point of $\{x_k\}$ if some subsequence converge to $x$. Recall that every bounded sequence in $\NDemenstionalRealEuclidianSpace$ converges to $x$ if and only if it is bounded and has $x$ as its unique cluster point.

  Let $\{x_k\}$ be a sequence in $\NDemenstionalRealEuclidianSpace$. We are interested in knowing how to handle convergence properties, we are led to consider direction $d_k \coloneqq x_k\left\lVert x_k\right\rVert^{-1}$ with $x_k \neq 0$, $k \in \NaturalNumberSet$. From classical analysis, the Bolzano-Weierstrass theorem implies that we can extract a convergent subsequence $d = \lim_{k \in K} d_k$, $K \subset \NaturalNumberSet$, with $d \neq 0$. Now suppose that the sequence $\{x_k\} \subset \NDemenstionalRealEuclidianSpace$ is such that $\left\lVert x_k \right\rVert \rightarrow + \infty$. Then
  \begin{equation}
    \exists t_k \coloneqq \left\lVert x_k \right\rVert, k \in K \subset \NaturalNumberSet, \text{ such that } \lim_{k \in K} t_k = + \infty \:\text{and}\: \lim_{k \in K} \frac{x_k}{t_k} = d. \notag
  \end{equation}
  This leads us to introduce the following concepts.
}

\DEFINITION{2.1.1}{
  A sequence $\{x_k\} \subset \RealNumberSet$ is said to converge to a direction $d \in \NDemenstionalRealEuclidianSpace$ if
  \begin{equation}
    \exists \{t_k\}, \:with\: t_k \rightarrow + \infty \:\text{ such that }\: \lim_{k \to \infty} \frac{x_k}{t_k} = d. \notag
  \end{equation}
}

\DEFINITION{2.1.2}{
  Let $C$ be a nonempty set in $\NDemenstionalRealEuclidianSpace$. Then the asymptotic cone of the set $C$, denoted by $C_{\infty}$, is the set of vectors $d \in \NDemenstionalRealEuclidianSpace$ that are limits in direction of the sequences $\{x_k\} \subset C$, namely
  \begin{equation}
    C_{\infty} = \{d \in \NDemenstionalRealEuclidianSpace \:|\: \exists t_k \rightarrow + \infty , \exists x_k \in C \:\text{ with }\: \lim_{k \to \infty} \frac{x_k}{t_k} = d \}. \notag
  \end{equation}
}

\QUOTEBOX{
  From the definition we immediately deduce the following elementary facts.
}

\PROPOSITION{2.1.1}{
  Let $C \subset \NDemenstionalRealEuclidianSpace$ be nonempty. Then:
  \begin{enumerate}[label=\roman*,align=CenterWithParen]
    \item $C_{\infty}$ is a closed cone.
    \item $(\text{cl}\:C)_{\infty} = C_{\infty}$.
    \item If $C$ is a cone, then $C_{\infty} = \text{cl}\:C$.
  \end{enumerate}
}

\begin{proof}
  We will prove each part separately.
  \begin{enumerate}[label=\roman*,align=CenterWithParen]
    \item $C_{\infty}$ is a closed cone.

      We need to show two propositions: (i-a) $C_{\infty}$ is a cone and (i-b) $C_{\infty}$ is a closed set.

      \begin{enumerate}[label=i-\alph*,align=CenterWithParen]
        \item We show that $C_{\infty}$ is a cone, that is, $\forall \alpha \geq 0, d \in C_{\infty}, \alpha d \in C_{\infty}$.

        Since $0$ is a element of $C_{\infty}$, it is clear in the case of $\alpha = 0$.

        ($\because$ Since $C$ is nonempty, we can take a element $x_0$ from $C$. In addition we take a sequence $\{t_k\}_{k=1}^{\infty}$ with $t_k \rightarrow + \infty$ as $k \rightarrow \infty$. Of course this sequence exists, for example $t_k \coloneqq k$. By using $t_k \coloneqq k$ and $x_k \coloneqq x_0$, we can obtain $0$ as the limit. Hence $0$ is a element of $C_{\infty}$.)

        Also we consider the other case $\alpha > 0$. To prove that $C_{\infty}$ is a cone, we take a any direction $d$ from $C_{\infty}$. Since d is a element of $C_{\infty}$,

        \begin{equation}
          \exists t_k \rightarrow + \infty , \exists x_k \in C \:\text{ with }\: \lim_{k \to \infty} \frac{x_k}{t_k} = d. \notag
        \end{equation}

        Then we define a sequence $\{t'_k\}_{k=1}^{\infty} \coloneqq \frac{t_k}{\alpha}$, exactly whose limit becomes $+\infty$ as $k \rightarrow \infty$. Accordingly there exist $t'_k \rightarrow + \infty$ and $x_k \in C$ with

        \begin{equation}
          \lim_{k \to \infty} \frac{x_k}{t'_k} = \lim_{k \to \infty} \alpha \cdot \frac{x_k}{t_k} = \alpha d. \notag
        \end{equation}

        This means $d \in C_{\infty}$.


        By these results, we can get $$\forall \alpha \geq 0, d \in C_{\infty}, \alpha d \in C_{\infty}$$. Therefore $C_{\infty}$ is a cone.

        \item We show that $C_{\infty}$ is a closed set. In order to prove closeness, we consider convergency of a sequence of $C_{\infty}$. First we take a sequence $\{d_k\}_{k=1}^{\infty} \subset C_{\infty}$ with $d_k \rightarrow d$ as $k \rightarrow \infty$ for some $d$. Then we don't forget that $d \in C_{\infty}$ is our goal.

        For each $d_k$,

        % add a figure about sequences
      \end{enumerate}

    \item $(\text{cl}\:C)_{\infty} = C_{\infty}$.

    \item If $C$ is a cone, then $C_{\infty} = \text{cl}\:C$.



  \end{enumerate}
\end{proof}

\QUOTEBOX{
  The importance of the asymptotic cone is revealed by the following key property, which is a immediate consequence of its definition.
}

\PROPOSITION{2.1.2}{
  A set $C \subset \NDemenstionalRealEuclidianSpace$ is bounded if and only if $C_{\infty} = \{0\}$.
}


\end{document}
