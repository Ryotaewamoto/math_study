\documentclass[a4paper,11pt]{jsarticle}

% 英文化
\usepackage[english]{babel}
% 数式
\usepackage{amsmath}
\usepackage{amsfonts}
\usepackage{amsthm}
\usepackage{amssymb}
\usepackage{bm}
\usepackage{mathtools}
% 画像
\usepackage[dvipdfmx]{graphicx}
% 箇条書き
\usepackage{enumitem}

% 枠付き文章
\usepackage{ascmac}

\DeclareFontFamily{U}{MnSymbolD}{}
\DeclareFontShape{U}{MnSymbolD}{m}{n}{
    <-6>  MnSymbolD5
   <6-7>  MnSymbolD6
   <7-8>  MnSymbolD7
   <8-9>  MnSymbolD8
   <9-10> MnSymbolD9
  <10-12> MnSymbolD10
  <12->   MnSymbolD12}{}
\DeclareFontShape{U}{MnSymbolD}{b}{n}{
    <-6>  MnSymbolD-Bold5
   <6-7>  MnSymbolD-Bold6
   <7-8>  MnSymbolD-Bold7
   <8-9>  MnSymbolD-Bold8
   <9-10> MnSymbolD-Bold9
  <10-12> MnSymbolD-Bold10
  <12->   MnSymbolD-Bold12}{}
\DeclareSymbolFont{MnSyD}{U}{MnSymbolD}{m}{n}
\SetSymbolFont{MnSyD}{bold}{U}{MnSymbolD}{b}{n}

\DeclareMathSymbol\preccurlyeq{\mathrel}{MnSyD}{"6C}
\DeclareMathSymbol\npreccurlyeq{\mathrel}{MnSyD}{"E4}

\newcommand{\MiniBox}[1]{\fbox{
  \begin{minipage}{.8\textwidth}
    #1
  \end{minipage}
}}

\SetLabelAlign{Center}{\hfil#1\hfil}
\SetLabelAlign{CenterWithParen}{\hfil(\makebox[1.0em]{#1})\hfil}

\newtheorem{thm}{Theorem}[section]
\newtheorem{prop}[thm]{Proposition}
\newtheorem{lem}[thm]{Lemma}
\newtheorem{cor}[thm]{Corollary}
\newtheorem{conj}[thm]{Conjecture}
\theoremstyle{definition}
\newtheorem{dfn}[thm]{Definition}
\newtheorem{rem}[thm]{Remark}
\newtheorem{fact}[thm]{Fact}

\renewcommand\labelenumi{(\theenumi)}

% Mathematical Sets
\newcommand{\NaturalNumberSet}{\mathbb{N}}
\newcommand{\RealNumberSet}{\mathbb{R}}
\newcommand{\NDemenstionalRealEuclideanSpace}{\mathbb{R}^n}
\newcommand{\MDemenstionalRealEuclideanSpace}{\mathbb{R}^m}
\newcommand{\NDemenstionalRealSymmetricMatrixSpace}{\mathbb{S}^n}
\newcommand{\NDemenstionalRealOthonormalMatrixSpace}{\mathcal{U}_n}


% Symbols like prefix
\newcommand{\Closure}[1]{\text{\rm cl\:${#1}$}} % cl
\newcommand{\Interior}[1]{\text{\rm int\:${#1}$}} % int
\newcommand{\Domain}[1]{\text{\rm dom\:${#1}$}} % dom
\newcommand{\Epigraph}[1]{\text{\rm epi\:${#1}$}} % epi
\newcommand{\KEpigraph}[1]{\text{\rm epi$_K$\:${#1}$}} % epi_K
\newcommand{\LevelSets}[2]{\text{\rm lev\:$({#1}, {#2})$}} % lev
\newcommand{\OrderingLevelSets}[3]{\text{\rm lev\:$({#1}, {#2}, {#3})$}} % lev
\newcommand{\Trace}[1]{\text{\rm tr$({#1})$}} % tr
\newcommand{\Diagnosis}[1]{\text{\rm diag\:${#1}$}} % diag
\newcommand{\InnerProduct}[2]{\left\langle {#1},{#2}\right\rangle} % <x,y>
\newcommand{\Norm}[1]{\left\lVert {#1} \right\rVert} % ||x||

% Extended real valued function e.g. f: X -> Rv{+∞}
% #1: function symbol
% #2: domain of function
\newcommand{\ExtendedRealValuedFunction}[2]{{#1}: {#2} \to \RealNumberSet \cup \{+\infty\}}
\newcommand{\VectorValuedFunction}[3]{{#1}\:\colon{#2} \to {#3}}


% Conjugate function e.g. f*
% #1: function symbol
\newcommand{\ConjugateFunction}[1]{{#1}^*}

% Support function e.g. f*
% #1: set symbol
\newcommand{\SupportFunction}[1]{\sigma_{#1}}

% Indicator function e.g. f*
% #1: set symbol
\newcommand{\IndicatorFunction}[1]{\delta_{#1}}

% (Useful) Texts
\newcommand{\SuchThat}{\:\text{\rm s.t.}\:}

% Set form e.g. {x | ...}
% #1: element
% #2: conditions
\newcommand{\SetForm}[2]{
  \{{#1}\:|\:{#2}\}
}

\begin{document}

\title{%
  Set-valued Ky Fan inequalities via scalarization}
\author{Ryota Iwamoto and Tamaki Tanaka}
\date{\today}
\maketitle

\section{Introduction}
In convex analysis and optimization theory, Ky Fan minimax inequality plays a key role. A quarter century ago, Georgiev and Tanaka \cite{GeorgievTanaka2000, GeorgievTanaka2001}
extended Ky Fan minimax inequality for set-valued maps. After that, Kuwano, Tanaka, and Yamada \cite{KuwanoTanakaYamada2010} constructed the result of four types set-valued Ky Fan minimax inequality
with set relations \cite{KuroiwaTanakaHa1997}, which are binary relations depending on a given convex cone. However, this result is limited to the case of the specific scalarization functions. To obtain more practical results,
we need to generalize the convexity properties for set-valued maps. In addition, Dechboon and Tanaka \cite{DechboonTanaka2024} proposed generalized continuity to inherit properties of cone continuity for set-valued maps.
The aim of this paper is to generalize the convexity properties for set-valued maps and to apply them to the set-valued Ky Fan minimax inequality.

\section{Mathematical Preliminaries}

Basically, let $X$ be a topological space, $Y$ a real topological vector space, and $\theta_Y$ be a zero vector in $Y$.
Define that $\mathcal{P}(Y)$ is the set of all nonempty subsets of $Y$. The sets of neighborhoods of $x \in X$ and $y \in Y$ is denoted by $\mathcal{N}_X (x)$ and $\mathcal{N}_Y (y)$, respectively.

\subsection{Set relations and these scalarization functions}

\begin{dfn}
  For $A,B \in \mathcal{P}(Y)$, we define two binary relations on $\mathcal{P}(Y)$:
  \begin{equation}
    A \preccurlyeq_1 B \overset{\text{def}}{\Longleftrightarrow} A \cap B \ne \emptyset \quad \text{and} \quad A \preccurlyeq_2 \overset{\text{def}}{\Longleftrightarrow} B \subset A. \notag
  \end{equation}
\end{dfn}

\subsection{Semicontinuity for set-valued maps}

\begin{dfn}[\cite{DechboonTanaka2024}]
  Let $F \colon X \to \mathcal{P}(Y)$, $x_0 \in X$, $\preccurlyeq$ a binary relation on $\mathcal{P}(Y)$
  and $C \subset Y$ a convex cone. We say that $F$ is $(\preccurlyeq, C)$-continuous at $x_0$ if
  \begin{equation}
    \forall W \subset Y, W\:\text{open},W \preccurlyeq F(x_0), \exists V \in \mathcal{N}_{X}(x_0) \SuchThat W + C \preccurlyeq F(x), \forall x \in V. \notag
  \end{equation}
\end{dfn}

\begin{dfn}[\cite{DechboonTanaka2024}]
  Let $\varphi \colon \mathcal{P}(Y) \to \RealNumberSet \cup \{\pm \infty\}$, $A_0 \in \mathcal{P}(Y)$,
  $\preccurlyeq$ a binary relation on $\mathcal{P}(Y)$, and $C$ a convex cone in $Y$ with $C \ne Y$. Then,
  we say that $\varphi$ is $(\preccurlyeq, C)$-lower semicontinuous at $A_0$ if
  \begin{equation}
    \forall r < \varphi (A_0), \exists W \in \mathcal{P}(Y), W\:\text{open}, \SuchThat W \preccurlyeq A_0 \:\text{and}\:
    r > \varphi (A), \forall A \in U(W + C, \preccurlyeq); \notag
  \end{equation}
  where $U(V,\preccurlyeq) \coloneqq \SetForm{A \in \mathcal{P}(Y)}{V \preccurlyeq A}$.
\end{dfn}

\begin{thm}[\cite{DechboonTanaka2024}]
  Let $F \colon X \to \mathcal{P}(Y)$, $\varphi \colon \mathcal{P}(Y) \to \RealNumberSet \cup \{\pm \infty\}$, $x_0 \in X$,
  $\preccurlyeq$ a binary relation on $\mathcal{P}(Y)$, and $C$ a convex cone. If $F$ is $(\preccurlyeq, C)$-continuous at $x_0$
  and $\varphi$ is $(\preccurlyeq, C)$-lower semicontinuous at $F(x_0)$, then $(\varphi \circ F)$ is lower semicontinuous at $x_0$.
\end{thm}

\begin{dfn}
  Let $\mathcal{A} \subset \mathcal{P}(Y) \backslash \{\emptyset\}$. $\mathcal{A}$ is said to be convex if for each $A_1, A_2 \in \mathcal{A}$ and $\lambda \in (0,1)$,
  \begin{equation}
    \lambda A_1 + (1-\lambda) A_2 \in \mathcal{A}. \notag
  \end{equation}
\end{dfn}

\begin{dfn}
  Let $\varphi \colon \mathcal{P}(Y) \to \RealNumberSet \cup \{\pm \infty\}$. Then,
  \begin{enumerate}
    \item $\varphi$ is quasi convex if for any $\alpha \in \RealNumberSet$,
          $\OrderingLevelSets{\varphi}{\leq}{\alpha} \coloneqq \SetForm{A \in \mathcal{P}(Y) \backslash \{\emptyset\}}{\varphi(A) \leq \alpha}$ is convex.
    \item $\varphi$ is quasi concave if for any $\alpha \in \RealNumberSet$,
          $\OrderingLevelSets{\varphi}{\geq}{\alpha} \coloneqq \SetForm{A \in \mathcal{P}(Y) \backslash \{\emptyset\}}{\varphi(A) \geq \alpha}$ is convex.
  \end{enumerate}
\end{dfn}

\subsection{Quasiconvexity properties for composite functions of set-valued map and scalarization function}

\begin{dfn}
  Let $X$ be a nonempty set, $Y$ a real topological vector space, $C$ a convex cone in $Y$, and $F\colon X \to 2^Y \backslash \{\emptyset\}$ a set-valued map.
  \begin{enumerate}
    \item $F$ is called $(\preccurlyeq)$-convex if for each $x,y \in X$ and $\lambda \in (0,1)$,
          \begin{equation}
            F(\lambda x + (1-\lambda)y) \preccurlyeq \lambda F(x) + (1-\lambda) F(y). \notag
          \end{equation}
    \item $F$ is called $(\preccurlyeq)$-properly quasi convex if for each $x,y \in X$ and $\lambda \in (0,1)$,
          \begin{equation}
            F(\lambda x + (1-\lambda)y) \preccurlyeq F(x) \quad \text{or} \quad F(\lambda x + (1-\lambda)y) \preccurlyeq F(y) \notag
          \end{equation}
    \item $F$ is called $(\preccurlyeq)$-naturally quasi convex if for each $x,y \in X$ and $\lambda \in (0,1)$, there exists $\mu \in [0,1]$ such that
          \begin{equation}
            F(\lambda x + (1-\lambda)y) \preccurlyeq \mu F(x) + (1-\mu) F(y). \notag
          \end{equation}
  \end{enumerate}
\end{dfn}

\begin{dfn}
  Let $X$ be a nonempty set, $Y$ a real topological vector space, $C$ a convex cone in $Y$, and $F\colon X \to 2^Y \backslash \{\emptyset\}$ a set-valued map.
  \begin{enumerate}
    \item $F$ is called $(\preccurlyeq)$-concave if for each $x,y \in X$ and $\lambda \in (0,1)$,
          \begin{equation}
            \lambda F(x) + (1-\lambda) F(y) \preccurlyeq F(\lambda x + (1-\lambda)y) . \notag
          \end{equation}
    \item $F$ is called $(\preccurlyeq)$-properly quasi concave if for each $x,y \in X$ and $\lambda \in (0,1)$,
          \begin{equation}
            F(x) \preccurlyeq F(\lambda x + (1-\lambda)y) \quad \text{or} \quad F(y) \preccurlyeq F(\lambda x + (1-\lambda)y) \notag
          \end{equation}
    \item $F$ is called $(\preccurlyeq)$-naturally quasi concave if for each $x,y \in X$ and $\lambda \in (0,1)$, there exists $\mu \in [0,1]$ such that
          \begin{equation}
            \mu F(x) + (1-\mu) F(y) \preccurlyeq F(\lambda x + (1-\lambda)y). \notag
          \end{equation}
  \end{enumerate}
\end{dfn}

\begin{rem}
  Obvously, if $F$ is $(\preccurlyeq)$-properly quasi convex, then $F$ is $(\preccurlyeq)$-properly quasi convex.
\end{rem}

\begin{thm}
  Let $\varphi$ be $(\preccurlyeq)$-monotone and $(\preccurlyeq)$-quasi convex. If $F$ is $(\preccurlyeq)$-naturally quasi convex, then $(\varphi \circ F)$ is quasi convex.
\end{thm}

\begin{thm}
  Let $\varphi$ be $(\preccurlyeq)$-monotone and $(\preccurlyeq)$-quasi concave. If $F$ is $(\preccurlyeq)$-naturally quasi concave, then $(\varphi \circ F)$ is quasi concave.
\end{thm}

\section{Scalarization functions preserving well properties}

To extend Ky Fan inequality for set-valued maps with a binary relation, consider assumptions of scalarization fucntions. To begin with, introduce four properties;
\begin{enumerate}
  \item $\varphi$ is $(\preccurlyeq, C)$-lower semicontinuous,
  \item $\varphi$ is quasi concave,
  \item $\varphi$ is $(\preccurlyeq)$-monotone,
  \item $\varphi(\{\theta\}) = 0$,
\end{enumerate}
and define the set of functions satisfying these properties as $\Phi(\preccurlyeq, C)$. In addition, establish three vaital properties for Ky Fan inequality;
\begin{enumerate}[label=(A\arabic*)]
  \item $\varphi (A) \leq 0 \Rightarrow A \preccurlyeq \{\theta\}$,
  \item there is an open neighborhood $G$ of $\theta$ such that $\{\theta\} + G \preccurlyeq A$, then
        $0 < \varphi (A)$,
  \item there is an open neighborhood $G$ of $\theta$ such that $\{\theta\} \preccurlyeq A + G$, then
        $0 < \varphi (A)$.
\end{enumerate}

\section{Applications for Ky-Fan Minimax Inequality}

Recall original Ky Fan inequality and provide main results

\begin{thm}[\cite{Fan1972}]
  Let $X$ be a nonempty compact convex subset of a Hausdorff topological vector space and $f \colon X \times X \to \RealNumberSet$. If $f$ satisfies
  the following conditions:
  \begin{enumerate}
    \item for each fixed $y \in X$, $f(\cdot,y)$ is lower semicontinuous,
    \item for each fixed $x \in X$,$f(x,\cdot)$ is quasi concave,
    \item $f(x,x) \leq 0$ for all $x \in X$,
  \end{enumerate}
  then there exists $\bar{x} \in X$ such that $f(\bar{x},y) \leq 0$ for all $y \in X$.
\end{thm}

\begin{thm}
  Let $X$ be a nonempty compact convex subset of a Hausdorff topological vector space,
  $Y$ a real topological vector space, $\preccurlyeq$ a binary relation on $\mathcal{P}(Y)$,
  $C$ a convex cone in $Y$, $\varphi\colon \mathcal{P}(Y) \to \RealNumberSet \cup \{\pm \infty\}$,
  and $F\colon X \times X \to \mathcal{P}(Y) \backslash \{\emptyset\}$ a set-valued map.
  For the scaralization function $\varphi \in \Phi(\preccurlyeq, C)$ satisfying Assumption (A1),
  if $F$ satisfies the following conditions:
  \begin{enumerate}
    \item $(\varphi \circ F)(x,y) \in \RealNumberSet$ for all $x,y \in X$,
    \item for each fixed $y \in X$, $F(\cdot,y)$ is $(\preccurlyeq, C)$-continuous,
    \item for each fixed $x \in X$, $F(x,\cdot)$ is $(\preccurlyeq)$-naturally quasi concave,
    \item for all $x \in X$, $F(x,x) \preccurlyeq \{\theta\}$,
  \end{enumerate}
  then there exists $\bar{x} \in X$ such that $ F(\bar{x},y) \preccurlyeq \{\theta\} $ for all $y \in X$.
\end{thm}

\begin{thm}
  Let $X$ be a nonempty compact convex subset of a Hausdorff topological vector space,
  $Y$ a real topological vector space, $\preccurlyeq$ a binary relation on $\mathcal{P}(Y)$,
  $C$ a convex cone in $Y$, $\varphi\colon \mathcal{P}(Y) \to \RealNumberSet \cup \{\pm \infty\}$,
  and $F\colon X \times X \to \mathcal{P}(Y) \backslash \{\emptyset\}$ a set-valued map.
  For the scaralization function $\varphi \in \Phi(\preccurlyeq, C)$ satisfying Assumption (A2),
  if $F$ satisfies the following conditions:
  \begin{enumerate}
    \item $(\varphi \circ F)(x,y) \in \RealNumberSet$ for all $x,y \in X$,
    \item for each fixed $y \in X$, $F(\cdot,y)$ is $(\preccurlyeq, C)$-continuous,
    \item for each fixed $x \in X$, $F(x,\cdot)$ is $(\preccurlyeq)$-naturally quasi concave,
    \item for all $x \in X$, $F(x,x) \preccurlyeq \{\theta\}$,
  \end{enumerate}
  then for any open neighborhood $G$ of $\theta$ there exists $\bar{x} \in X$ such that $\{\theta\} + G \npreccurlyeq F(\bar{x},y)$
  for all $y \in X$.
\end{thm}

\begin{thm}
  Let $X$ be a nonempty compact convex subset of a Hausdorff topological vector space,
  $Y$ a real topological vector space, $\preccurlyeq$ a binary relation on $\mathcal{P}(Y)$,
  $C$ a convex cone in $Y$, $\varphi\colon \mathcal{P}(Y) \to \RealNumberSet \cup \{\pm \infty\}$,
  and $F\colon X \times X \to \mathcal{P}(Y) \backslash \{\emptyset\}$ a set-valued map.
  For the scaralization function $\varphi \in \Phi(\preccurlyeq, C)$ satisfying Assumption (A3),
  if $F$ satisfies the following conditions:
  \begin{enumerate}
    \item $(\varphi \circ F)(x,y) \in \RealNumberSet$ for all $x,y \in X$,
    \item for each fixed $y \in X$, $F(\cdot,y)$ is $(\preccurlyeq, C)$-continuous,
    \item for each fixed $x \in X$, $F(x,\cdot)$ is $(\preccurlyeq)$-naturally quasi concave,
    \item for all $x \in X$, $F(x,x) \preccurlyeq \{\theta\}$,
  \end{enumerate}
  then for any open neighborhood $G$ of $\theta$ there exists $\bar{x} \in X$ such that
  $\{\theta\} \npreccurlyeq F(\bar{x},y) + G$ for all $y \in X$.
\end{thm}

\begin{thebibliography}{99}
  \bibitem{DechboonTanaka2024}
  P. Dechboon and T. Tanaka, Inheritance Properties on Cone Continuity for Set-Valued Maps via Scalarization, Minimax Theory and its Applications. 9 (2024),

  \bibitem{Fan1972}
  K. Fan, A minimax inequality and its applications, Inequalities III, O. Shisha (ed.), Academic Press, New York, (1972), 103--113.

  \bibitem{GeorgievTanaka2000}
  P. G. Georgiev and T. Tanaka, Vector-valued set-valued variants of Ky Fan's inequality, J. Nonlinear and Convex Anal. 1 (2000), 245--254.

  \bibitem{GeorgievTanaka2001}
  P. G. Georgiev and T. Tanaka, Fan's inequality for set-valued maps, Nonlinear Anal. 47 (2001), no.1, 607--618.

  \bibitem{KobayashiSaitoTanaka2016}
  S. Kobayashi, Y. Saito, and T. Tanaka, Convexity for compositions of set-valued map and monotone scalarizing function, Yokohama Publishers, Yokohama, (2016), 43--54.

  \bibitem{KuroiwaTanakaHa1997}
  D. Kuroiwa, T. Tanaka, and T.X.D. Ha, On cone convexity of set-valued maps, Nonlinear Anal. 30 (1997), 1487--1496.

  \bibitem{KuwanoTanakaYamada2010}
  I. Kuwano, T. Tanaka, and S. Yamada, Unified scalarization for sets and set-valued Ky Fan minimax inequality, J. Nonlinear Convex Anal. 11 (2010), 513--525.

  \bibitem{SondaKuwanoTanaka2010}
  Y. Sonda, I. Kuwano, and T. Tanaka, Cone-semicontinuity of set-valued maps by analogy with real-valued semicontinuity, Nihonkai Mathematical Journal. 21 (2010), 91--103.

\end{thebibliography}

\end{document}