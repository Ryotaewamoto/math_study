%%% Template LaTeX file for JSRC2017
\documentclass[a4,11pt]{amsart}
\usepackage{amsmath,amssymb,amsthm}
\addtolength{\textwidth}{100pt}
\addtolength{\oddsidemargin}{-60pt}
\addtolength{\textheight}{120pt}
\addtolength{\topmargin}{-70pt}
%
\usepackage[dvipdfmx]{graphicx}
%\usepackage{graphicx}
\renewcommand{\thepage}{\empty}
%
\begin{document}
\pagestyle{plain}
%
%%%%%%%%%%%%%%%%%%%%%%%%%%%%%%%%%%%%%%%%%%%%%%%%%%%%%%%%%%%%%%%%%%%%%%%%%%
\title[A RELATION BETWEEN ASYMPTOTIC CONES AND PAINLEV\'E-KUROTOWSKI CONVERGENCE]%please input abbreviation of the title in []
{A RELATION BETWEEN ASYMPTOTIC CONES AND PAINLEV\'E-KUROTOWSKI CONVERGENCE}
\author[RYOTA IWAMOTO, TAMAKI TANAKA]%please input abbreviation of the authors' names in []
{RYOTA IWAMOTO, TAMAKI TANAKA
\\
Niigata University, Japan}
\thanks{*Presenting author.}
\address[R.Iwamoto, T.Tanaka]%please input abbreviation of the authors' names in []
{Niigata University, Niigata 950–2181, Japan.}
\email{f22a034c@mail.cc.niigata-u.ac.jp}
%
%\keywords{}
%\subjclass[2000]{Primary ; Secondary }
%
%%%%%%%%%%%%%%%%%%%%%%%%%%%%%%%%%%%%%%%%%%%%%%%%%%%%%%%%%%%%%%%%%%%%
\maketitle
%%%%%%%%%%%%%%%%%%%%%%%%%%%%%%%%%%%%%%%%%%%%%%%%%%%%%%%%%%%%%%%%%%%%%
%% Please input your abstracts here %%
A asymptotic cone is a cone of a set and is considered about the limit of a sequence in the set. This cone has a lot of useful properties, in addition, a cone itself is often used such as a optimization or a set-relation in the field of convex analysis.

Painlev\'e-Kuratowski Convergence is a notion in topological vector space and implies the convergence of set-valued mapping. Characterizing the definition of asymptotic cone with this convergence, asymptotic cone can be redefined as set-valued mapping.

This presentation gives a relation between asymptotic cones and Painlev\'e-Kuratowski Convergence on convex analysis. The former part explains what asymptotic cones is and what Painlev\'e-Kuratowski Convergence means. The latter part introduces that asymptotic cones can be written as the set-valued mapping indeed.
%%%%%%%%%%%%%%%%%%%%%%%%%%%%%%%%%%%%%%%%%%%%%%%%%%%%%%%%%%%%%%%%%%%%%%
\begin{thebibliography}{9}%% If you want to make "Reference", please input here.
\bibitem{} A. Gopfert, H. Riahi, C. Tammer, and C. Zalinescu, Variational methods in partially ordered spaces, vol. 17 of CMS Books in Mathematics, Springer-Verlag, New York, 2003.
\bibitem{} A. Alfred and M. Teboulle, asymptotic cones and functions in optimization and variational inequalities, Springer monographs in Mathematics, Springer-Verlag, New York, 2003.
\end{thebibliography}
%%%%%%%%%%%%%%%%%%%%%%%%%%%%%%%%%%%%%%%%%%%%%%%%%%%%%%%%%%%%%%%%%%%%%%
%
\end{document}


