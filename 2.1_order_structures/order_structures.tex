% !TeX root = definitions_of_asymptotic_cones.tex
\documentclass[a4paper,11pt]{jsarticle}

% use English
\usepackage[english]{babel}
% mathematical expression or mathematical formula
\usepackage{amsmath}
\usepackage{amsfonts}
\usepackage{amsthm}
\usepackage{amssymb}
\usepackage{bm}
\usepackage{mathtools}
% images
\usepackage[dvipdfmx]{graphicx}
% bullet points
\usepackage{enumitem}
% boxes with lines
\usepackage{ascmac}

% Boxes
\newcommand{\QUOTEBOX}[1]{\begin{center}\fbox{\begin{minipage}{.8\textwidth}{#1}\end{minipage}}\end{center}} % quote
\newcommand{\DEFINITION}[2]{\begin{itembox}[l]{\underline{Definition {#1} }}{#2}\end{itembox}} % definition
\newcommand{\PROPOSITION}[2]{\begin{itembox}[l]{\underline{Proposition {#1} }}{#2}\end{itembox}} % propositio
\newcommand{\NOTE}[1]{\begin{itembox}[l]{\underline{Note}}{#1}\end{itembox}} % note
\newcommand{\REMARK}[2]{\begin{itembox}[l]{\underline{Remark {#1} }}{#2}\end{itembox}} % remark

% (Useful) Sets
\newcommand{\NaturalNumberSet}{\mathbb{N}}
\newcommand{\RealNumberSet}{\mathbb{R}}
\newcommand{\NDemenstionalRealEuclidianSpace}{\mathbb{R}^n}

% (Useful) Texts
\newcommand{\SuchThat}{\:\text{s.t.}\:}
\newcommand{\Backslash}{\:\backslash\:}

% Symbols
\newcommand{\Int}[1]{\text{int\:${#1}$}} % int
\newcommand{\Cl}[1]{\text{cl\:${#1}$}} % cl
\newcommand{\Dom}[1]{\text{dom\:${#1}$}} % dom

% Set form e.g. {x | ...}
% #1: element
% #2: conditions
\newcommand{\SetForm}[2]{
  \{{#1}\:|\:{#2}\}
}

% Space e.g. (X,a)
% #1: set
% #2: topology
\newcommand{\Space}[2]{
  \text{(${#1}$, ${#2}$)}
}

% Structure e.g. (X,a)
% #1: set
% #2: structure
\newcommand{\Structure}[2]{
  \text{(${#1}$, ${#2}$)}
}

% Non-center neighborhood
% #1: set
\newcommand{\NonCenterNeighborhood}[1]{
  {#1}^{\bullet}
}

% Family of neighborhoods
% #1: based space
% #2: based point
\newcommand{\FammilyOfNeighborhood}[2]{
  \mathcal{V}_{{#1}}({#2})
}

% Multifunction
% #1: function name
% #2: domain
% #3: codomain
\newcommand{\Multifuncion}[3]{
  {#1} : {#2} \rightrightarrows {#3}
}

% Inverse sets (to function)
% #1: function name
% #2: domain
\newcommand{\ImageSet}[2]{
  {#1}({#2})
}

% Inverse sets (to function)
% #1(optional): inverse symbol (default=-)
% #2: function name
% #3: codomain
\newcommand{\InverseSet}[3][-]{
  {#2}^{{#1}1}({#3})
}

\SetLabelAlign{Center}{\hfil#1\hfil}
\SetLabelAlign{CenterWithParen}{\hfil(\makebox[1.0em]{#1})\hfil}
\SetLabelAlign{CenterWithParen2}{\hfil(\makebox[1.5em]{#1})\hfil}

\renewcommand{\theenumi}{\roman{enumi}}
\renewcommand{\labelenumi}{(\theenumi)}
\renewcommand{\theenumii}{\theenumi-\alph{enumii}}
\renewcommand{\labelenumii}{(\theenumii)}

\begin{document}

\title{
  2 Functional Analysis over Cones \\
  \large 2.1 Order Structures}
\author{Ryota Iwamoto}
\date{\today}
\maketitle

We use the book; Variational Methods in Partially Ordered Spaces (author: A.Gopfert, H.Riahi, C.Tammer, and C.Zalinescu).

p.13

\QUOTEBOX{
  As seen in the introduction, we are concerned with certain sets $M$ with order structures. In the sequel we give the basic definitions.
  As usual, when $M$ is a nonempty set, $M \times M$ is the set of ordered paris elemennts of $M$:
  \begin{equation}
    M \times M \coloneqq \SetForm{(x_1, x_2)}{x_1, x_2 \in M}. \notag
  \end{equation}
}

\DEFINITION{2.1.1}{
  Let $M$ be a nonempty set and $\mathcal{R}$ a nonempty subset of $M \times M$. Then $\mathcal{R}$ is called a binary relation or an order structure on $M$, and $\Structure{X}{\mathcal{R}}$ is a set $M$ with order structure $\mathcal{R}$. The fact that $(x_1,x_2) \in \mathcal{R}$ will be denoted by $x_1 \mathcal{R} x_2$. We say that $\mathcal{R}$ is

  \begin{enumerate}[label=\alph*,align=CenterWithParen]
    \item reflexive if $\forall x \in M : x \mathcal{R} x$,
    \item transitive if$\forall x_1, x_2, x_3 \in M : x_1 \mathcal{R} x_2, x_2 \mathcal{R} x_3 \Rightarrow x_1 \mathcal{R} x_3$,
    \item antisymmetric if $\forall x_1, x_2 \in M : x_1 \mathcal{R} x_2, x_2 \mathcal{R} x_1 \Rightarrow x_1 = x_2$.
  \end{enumerate}
}


\end{document}
