% !TeX root = definitions_of_asymptotic_cones.tex
\documentclass[a4paper,11pt]{jsarticle}

% use English
\usepackage[english]{babel}
% mathematical expression or mathematical formula
\usepackage{amsmath}
\usepackage{amsfonts}
\usepackage{amsthm}
\usepackage{amssymb}
\usepackage{bm}
\usepackage{mathtools}
% images
\usepackage[dvipdfmx]{graphicx}
% bullet points
\usepackage{enumitem}
% boxes with lines
\usepackage{ascmac}

% Boxes
\newcommand{\QUOTEBOX}[1]{\begin{center}\fbox{\begin{minipage}{.8\textwidth}{#1}\end{minipage}}\end{center}} % quote
\newcommand{\DEFINITION}[2]{\begin{itembox}[l]{\underline{Definition {#1} }}{#2}\end{itembox}} % definition
\newcommand{\PROPOSITION}[2]{\begin{itembox}[l]{\underline{Proposition {#1} }}{#2}\end{itembox}} % propositio
\newcommand{\NOTE}[1]{\begin{itembox}[l]{\underline{Note}}{#1}\end{itembox}} % note
\newcommand{\REMARK}[2]{\begin{itembox}[l]{\underline{Remark {#1} }}{#2}\end{itembox}} % remark

% (Useful) Sets
\newcommand{\NaturalNumberSet}{\mathbb{N}}
\newcommand{\RealNumberSet}{\mathbb{R}}
\newcommand{\NDemenstionalRealEuclidianSpace}{\mathbb{R}^n}

% (Useful) Texts
\newcommand{\SuchThat}{\:\text{s.t.}\:}
\newcommand{\Backslash}{\:\backslash\:}

% Symbols
\newcommand{\Int}[1]{\text{int\:${#1}$}} % int
\newcommand{\Cl}[1]{\text{cl\:${#1}$}} % cl
\newcommand{\Dom}[1]{\text{dom\:${#1}$}} % dom

% Set form e.g. {x | ...}
% #1: element
% #2: conditions
\newcommand{\SetForm}[2]{
  \{{#1}\:|\:{#2}\}
}

% Space e.g. (X,a)
% #1: set
% #2: topology
\newcommand{\Space}[2]{
  \text{(${#1}$, ${#2}$)}
}

% Non-center neighborhood
% #1: set
\newcommand{\NonCenterNeighborhood}[1]{
  {#1}^{\bullet}
}

% Family of neighborhoods
% #1: based space
% #2: based point
\newcommand{\FammilyOfNeighborhood}[2]{
  \mathcal{V}_{{#1}}({#2})
}

% Multifunction
% #1: function name
% #2: domain
% #3: codomain
\newcommand{\Multifuncion}[3]{
  {#1} : {#2} \rightrightarrows {#3}
}

% Inverse sets (to function)
% #1: function name
% #2: domain
\newcommand{\ImageSet}[2]{
  {#1}({#2})
}

% Inverse sets (to function)
% #1(optional): inverse symbol (default=-)
% #2: function name
% #3: codomain
\newcommand{\InverseSet}[3][-]{
  {#2}^{{#1}1}({#3})
}

\SetLabelAlign{Center}{\hfil#1\hfil}
\SetLabelAlign{CenterWithParen}{\hfil(\makebox[1.0em]{#1})\hfil}
\SetLabelAlign{CenterWithParen2}{\hfil(\makebox[1.5em]{#1})\hfil}

\renewcommand{\theenumi}{\roman{enumi}}
\renewcommand{\labelenumi}{(\theenumi)}
\renewcommand{\theenumii}{\theenumi-\alph{enumii}}
\renewcommand{\labelenumii}{(\theenumii)}

\begin{document}

\title{
  2 Functional Analysis over Cones \\
  \large 2.5 Continuity Notions of Multifunctions}
\author{Ryota Iwamoto}
\date{\today}
\maketitle

We use the book; Variational Methods in Partially Ordered Spaces (author: A.Gopfert, H.Riahi, C.Tammer, and C.Zalinescu).

p.51

\QUOTEBOX{
  In this section $X$ and $Y$ are separated (in the sense of Hausdorff) topological spaces and $\Multifuncion{\varGamma}{X}{Y}$ a multifunction. When mentioned explicitly, $Y$ is a separeated topological vector space (s.t.v.s).
}

``$\rightrightarrows$'' is one of symbols that mean multifunction. Also, ``$\twoheadrightarrow $'' and ``$X \rightarrow 2^X$'' have the same meaning.

\DEFINITION{2.5.1}{
  Let $x_0 \in X$. We say that

  \begin{enumerate}[label=(\alph*)]
    % upper continuous at x_0
    \item $\varGamma$ is upper continuous (u.c.) at $x_0$ if
    \begin{equation}
      \forall D \subset Y, D \:\text{open}, \varGamma (x_0) \subset D, \exists U \in \FammilyOfNeighborhood{X}{x_0} \SuchThat \forall x \in U, \varGamma (x) \subset D, \tag*{(2.33)}
    \end{equation}

    i.e., $\InverseSet[+]{\varGamma}{D}$ is a neighborhood of $x_0$ for each open set $D \subset Y$ such that $F(x_0) \subset D$;

    % lower continuous at x_0
    \item $\varGamma$ is lower continuous (l.c.) at $x_0$ if
    \begin{equation}
      \forall D \subset Y, D \:\text{open}, \varGamma (x_0) \cap D \ne \emptyset , \exists U \in \FammilyOfNeighborhood{X}{x_0} \SuchThat \forall x \in U, \varGamma (x) \cap D \ne \emptyset, \tag*{(2.34)}
    \end{equation}

    i.e., $\InverseSet{\varGamma}{D}$ is a neighborhood of $x_0$ for each open set $D \subset Y$ such that $\varGamma (x) \cap D \ne \emptyset$.

    % continuous
    \item $\varGamma$ is continuous at $x_0$ if $\varGamma$ is u.c. and l.c. at $x_0$.

    % continuous everywhere
    \item $\varGamma$ is upper continuous (lower continuous, continuous) at $x_0$ if $\varGamma$ is so at every $x \in X$;

    % lower continuous at (x_0, y_0)
    \item $\varGamma$ is lower continuous at $(x_0, y_0) \in X \times Y$ if
    \begin{equation}
      \forall V \in \FammilyOfNeighborhood{Y}{y_0}, \exists U \in \FammilyOfNeighborhood{X}{x_0} \SuchThat \forall x \in U, \varGamma (x) \cap D \ne \emptyset. \notag
    \end{equation}

  \end{enumerate}
}

\QUOTEBOX{
  It follows from the definition that $x_0 \in \Int{(\Dom{\varGamma})}$ and $y_0 \in \Cl{(\varGamma (x_0))}$ if $\varGamma$ is l.c. at $(x_0, y_0)$ and $\varGamma$ is l.c. at $x_0 \in \Dom{\varGamma}$ if and only if $\varGamma$ is l.c. at every $(x_0, y_0)$ with $y \in \varGamma(x_0)$; moreover, $\varGamma$ is l.c. at every $x_0 \in X \Backslash \Dom{\varGamma}$. If $x_0 \in X \Backslash \Dom{\varGamma}$, then $\varGamma$ is u.c. at $x_0$ if and only if $x_0 \in \Int{(X \Backslash \Dom{\varGamma})}$. So, if $\varGamma$ is u.c., then $\Dom{\varGamma}$ is closed, while if $\varGamma$ is l.c., then $\Dom{\varGamma}$ is open.
  The next result follows immediately from the definitions.
}

It means that below.

\NOTE{
  \begin{enumerate}[label=\roman*,align=CenterWithParen]
    \item $\varGamma$: l.c. at $(x_0, y_0)$ $\Rightarrow$ $x_0 \in \Int{(\Dom{\varGamma})}$, $y_0 \in \Cl{(\varGamma (x_0))}$
    \item $\varGamma$: l.c. at $x_0 \in \Dom{\varGamma}$ $\Leftrightarrow$ $\varGamma$: l.c. at $\forall (x_0, y)$ with $y \in \varGamma (x_0)$
    \item $\varGamma$: l.c. at $\forall x_0 \in X \Backslash \Dom{\varGamma}$
    \item $\varGamma$: u.c. at $x_0 \in X \Backslash \Dom{\varGamma}$ $\Leftrightarrow$ $x_0 \in \Int{(X \Backslash \Dom{\varGamma})}$
    \item $\varGamma$: u.c. $\Rightarrow$ $\Dom{\varGamma}$: closed
    \item $\varGamma$: l.c. $\Rightarrow$ $\Dom{\varGamma}$: open
  \end{enumerate}
}

\begin{proof}
  Coming soon...
\end{proof}

\PROPOSITION{2.5.2}{
  \begin{enumerate}[label=\roman*,align=CenterWithParen]
    \item $\varGamma$: u.c. $\Leftrightarrow$ $\forall D \subset Y$: open, $\InverseSet[+]{\varGamma}{D}$: open
    \item $\varGamma$: l.c. $\Leftrightarrow$ $\forall D \subset Y$: open, $\InverseSet{\varGamma}{D}$: open
  \end{enumerate}
}

\begin{proof}
  Coming soon...
\end{proof}
\DEFINITION{(limit inferior and limi superior)}{
  The limit inferior of $\varGamma$ at $x_0 \in X$ is defeined by
  \begin{equation}
    \liminf_{x \to x_0} \varGamma (x) \coloneqq \SetForm{y \in Y}{\forall V \in \FammilyOfNeighborhood{Y}{y}, \exists U \in \FammilyOfNeighborhood{X}{x_0} \SuchThat \forall x \in \NonCenterNeighborhood{U}, \varGamma (x) \cap D \ne \emptyset}, \notag
  \end{equation}

  while the limit superior of $\varGamma$ at $x_0 \in X$ is defined by
  \begin{equation}
    \begin{split}
      \limsup_{x \to x_0} \varGamma (x) &\coloneqq \SetForm{y \in Y}{\forall V \in \FammilyOfNeighborhood{Y}{y}, \forall U \in \FammilyOfNeighborhood{X}{x_0}, \exists x \in \NonCenterNeighborhood{U} \SuchThat \varGamma (x) \cap D \ne \emptyset}, \\
      &= \bigcap_{U \in \FammilyOfNeighborhood{X}{x_0}} \Cl{(\ImageSet{\varGamma}{\NonCenterNeighborhood{U}})}, \notag
    \end{split}
  \end{equation}

  where for $U \in \FammilyOfNeighborhood{X}{x_0}$, $\NonCenterNeighborhood{U} \coloneqq U \Backslash \{x_0\}$.
}

\end{document}
