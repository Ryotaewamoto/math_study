\documentclass[a4paper,11pt]{jsarticle}


% 数式
\usepackage{amsmath,amsfonts, amsthm, amssymb}
\usepackage{bm}
\usepackage{mathtools}
% 画像
\usepackage[dvipdfmx]{graphicx}

\theoremstyle{definition}
\newtheorem{thm}{定理}
\newtheorem{prop}[thm]{命題}
\newtheorem{lem}[thm]{Lemma}
\newtheorem{cor}[thm]{Corollary}
\newtheorem{dfn}[thm]{定義}
\newtheorem{rem}[thm]{Remark}
\newtheorem{fact}[thm]{Fact}

% Mathematical Sets
\newcommand{\NaturalNumberSet}{\mathbb{N}}
\newcommand{\RealNumberSet}{\mathbb{R}}
\newcommand{\NDemenstionalRealEuclideanSpace}{\mathbb{R}^n}
\newcommand{\NDemenstionalRealSymmetricMatrixSpace}{\mathbb{S}^n}
\newcommand{\NDemenstionalRealOthonormalMatrixSpace}{\mathcal{U}_n}

% Symbols like prefix
\newcommand{\Closure}[1]{\text{\rm cl\:${#1}$}} % cl
\newcommand{\Interior}[1]{\text{\rm int\:${#1}$}} % int
\newcommand{\Domain}[1]{\text{\rm dom\:${#1}$}} % dom
\newcommand{\Epigraph}[1]{\text{\rm epi\:${#1}$}} % epi
\newcommand{\LevelSets}[2]{\text{\rm lev\:$({#1}, {#2})$}} % lev
\newcommand{\Trace}[1]{\text{\rm tr$({#1})$}} % tr
\newcommand{\Diagnosis}[1]{\text{\rm diag\:${#1}$}} % diag
\newcommand{\InnerProduct}[2]{\left\langle {#1},{#2}\right\rangle} % <x,y>
\newcommand{\Norm}[1]{\left\lVert {#1} \right\rVert} % ||x||

% Extended real valued function e.g. f: X -> Rv{+∞}
% #1: function symbol
% #2: domain of function
\newcommand{\ExtendedRealValuedFunction}[2]{{#1}: {#2} \to \RealNumberSet \cup \{+\infty\}}

% Conjugate function e.g. f*
% #1: function symbol
\newcommand{\ConjugateFunction}[1]{{#1}^*}

% Support function e.g. f*
% #1: set symbol
\newcommand{\SupportFunction}[1]{\sigma_{#1}}

% Indicator function e.g. f*
% #1: set symbol
\newcommand{\IndicatorFunction}[1]{\delta_{#1}}

% (Useful) Texts
\newcommand{\SuchThat}{\:\text{\rm s.t.}\:}

% Set form e.g. {x | ...}
% #1: element
% #2: conditions
\newcommand{\SetForm}[2]{
  \{{#1}\:|\:{#2}\}
}

\begin{document}

\title{漸近関数の半正定値計画問題への応用}
\author{F22A034C 岩本 崚汰}
\date{\vspace{-0ex}}

\maketitle

漸近関数は、最適化問題の解を求めるための有力なツールの一つである。この概念の起源は、1913年の Steinitz にあり、その後、 Stocker によって研究が進められた。その20年後、Bourbaki と Choquet によってさらに研究が進められ、1970年、Rockafellar によって凸集合に対しての asymptotic cone を recession cone として定義された。同様に、asymptotic function も recession function として定義され、この理論の基礎となる理論がまとめられた。1999年には、Auslender と Teboulle はこの漸近錐 (asymptotic cone) と漸近関数 (asymptotic function) の理論と応用を『Asymptotic cones and functions in optimization and variational inequalities』でまとめた。この本に記述されているように、漸近関数の応用の一つとして、半正定値計画問題との関係が挙げられる。半正定値計画問題は、線形計画問題、2次凸計画問題、二次錐計画問題の一般化であり、その応用は、組合せ最適化、グラフ理論、統計学、量子力学、機械学習など多岐にわたる。そのため、様々な問題に対して適応できることから近年注目を集めている。

はじめに、漸近錐と漸近関数の定義と簡単な性質を紹介する。

\begin{dfn}
  $C \subset \NDemenstionalRealEuclideanSpace$、$C \neq \emptyset$ とする。このとき$C$ の漸近錐 (Asymptotic cone)、記号で $C_\infty$、は点列 $\{ x_k \} \subset C$ を用いて以下のように定義する。

  \begin{equation}
    C_\infty = \left\{ d \in
    \mathbb{R}^n \:\middle|\: \exists t_k \rightarrow +\infty, \exists x_k \in C \:\text{\rm with}\: \lim_{k \to \infty} \frac{x_k}{t_k} = d \right\}. \notag
  \end{equation}
\end{dfn}

\begin{prop}
  $C \subset \NDemenstionalRealEuclideanSpace$、$C \neq \emptyset$ とする。この時、以下が成り立つ。
  \begin{enumerate}
    \item $C_{\infty}$ は閉凸錐
    \item $(\text{cl}\:C)_{\infty} = C_{\infty}$
    \item $C$ が錐ならば, $C_{\infty} = \text{cl}\:C$
  \end{enumerate}
\end{prop}

\begin{dfn}
  proper な関数 $\ExtendedRealValuedFunction{f}{\NDemenstionalRealEuclideanSpace}$ に対して, ただ一つの$f$に関する$f_{\infty}: \NDemenstionalRealEuclideanSpace \to \overline{\RealNumberSet}$を$\Epigraph{f_\infty} = (\Epigraph{f})_{\infty}$を満たすように定義する。この関数を$f$の漸近関数 (Asymptotic function)と呼ぶ。
\end{dfn}

\begin{prop}\label{basicPropositionOfAsymptoticFunctions}
  proper な関数 $\ExtendedRealValuedFunction{f}{\NDemenstionalRealEuclideanSpace}$ に対して以下が成り立つ。
  \begin{enumerate}
    \item $f_{\infty}$ は下半連続かつ positively homogeneous
    \item $f_{\infty}(0) = 0$ または $f_{\infty}(0) = - \infty$
    \item $f_{\infty}(0) = 0$ ならば, $f_{\infty}$ は proper
    \item $(\alpha f)(x) \coloneqq \alpha f(x)$, $\alpha > 0$ とする。このとき、$(\alpha f)_{\infty} = \alpha f_{\infty}$
  \end{enumerate}
\end{prop}

\begin{prop}
  $\ExtendedRealValuedFunction{f}{\NDemenstionalRealEuclideanSpace}$ を proper かつ下半連続な凸関数とする。このとき、この漸近関数 $f_{\infty}$ は positively homogeneous かつ proper で下半連続な凸関数であり、任意の $d \in \NDemenstionalRealEuclideanSpace$ に対して, 以下が成り立つ。
  \begin{equation}
    f_{\infty}(d) = \sup \{f(x+d) -f(x) \:|\: x \in \Domain{f}\}
  \end{equation}
  また、
  \begin{equation}
    f_{\infty}(d) = \lim_{t \to +\infty} \frac{f(x+td) -f(x)}{t} = \sup_{t>0} \frac{f(x+td) - f(x)}{t}, \forall x \in \Domain{f}
  \end{equation}
\end{prop}

次に半正定値計画問題への応用を考えるために、関数が symmetric であること、また spectrally defined であることを定義する。

\begin{dfn}
  $\ExtendedRealValuedFunction{f}{\NDemenstionalRealEuclideanSpace}$ とする。このとき、$f$ が symmetric であるとは、以下が成り立つことをいう。
  \begin{equation}
    \forall x \in \NDemenstionalRealEuclideanSpace \:\text{and}\: P: n \times n \text{ の置換行列},\quad f(Px) = f(x). \notag
  \end{equation}
\end{dfn}

以下が symmetric な関数の例である。

\begin{enumerate}
  \item $f(x) = \max_{1 \leq i \leq n} x_i$ (or $\min_{1 \leq i \leq n} x_i$),
  \item $f(x) = \sum_{i = 1}^{n} x_i$ (or $\prod_{i = 1}^{n} x_i$).
\end{enumerate}

\begin{dfn}
  関数 $\ExtendedRealValuedFunction{\Phi}{\NDemenstionalRealSymmetricMatrixSpace}$ が spectrally defined であるとは、以下を満たすような symmetric な関数 $\ExtendedRealValuedFunction{f}{\NDemenstionalRealEuclideanSpace}$ が存在することをいう。
  \begin{equation}
    \Phi (X) = \Phi_{f}(X) \coloneqq f(\lambda (X)), \forall X \in \NDemenstionalRealSymmetricMatrixSpace \notag
  \end{equation}
  ただし、$\lambda (X) \coloneqq (\lambda_1 (X), \dotsb , \lambda_n (X))^T$ は $X$ の固有値を非減少に並べたベクトルである。
\end{dfn}

\begin{thm}[\mbox{\cite[Lewis (1996)]{Lewis96}}]\label{Lewis96}
  $\ExtendedRealValuedFunction{f}{\NDemenstionalRealEuclideanSpace}$ は symmetric な関数とする。このとき、以下が成り立つ。
  \begin{equation}
    {\Phi_{f}}^* = \Phi_{f^*}. \notag
  \end{equation}
\end{thm}

\begin{thm}[\mbox{\cite[Seeger (1997)]{Seeger97}}]\label{Seeger97}
  $\ExtendedRealValuedFunction{f}{\NDemenstionalRealEuclideanSpace}$ はsymmetric かつ proper で下半連続な凸関数とし、その spectrally defined な関数を $\Phi_{f}$ とする。このとき、以下が成り立つ。
  \begin{equation}
    {(\Phi_{f})}_{\infty} = \Phi_{f_{\infty}}. \notag
  \end{equation}
\end{thm}

\begin{thebibliography}{99}
  \bibitem{Auslender99}  A. Auslender. Penalty and barrier methods: A unified framework. SIAM J. Optimization, 10 (1999), 211--230.

  \bibitem{Auslender03}
  A. Auslender and M. Teboulle. Asymptotic cones and functions in optimization and variational inequalities, Springer monographs in Mathematics, Springer-Verlag, New York, 2003.

  \bibitem{BeckTeboulle12}
  A. Beck and M. Teboulle. Smoothing and First Order Methods: A Unified Framework, SIAM J. Optim, 22 (2012), 557--580.

  \bibitem{BenTalTeboulle89}
  A. Ben-Tal and M. Teboulle. A smoothing technique for nondifferentiable optimization problems. In Optimization, Fifth French-German Conference, Lecture Notes in Mathematics 1405, Springer-Verlag, New York (1989), 1--11.

  \bibitem{BorweinLewis00}
  J.M. Borwein and A.S. Lewis. Convex Analysis and Nonlinear Optimization: Theory and Examples, Springer-Verlag, New York, 2000.

  \bibitem{Lewis96}
  A.S. Lewis. Convex Analysis on the Hermitian matrices. SIAM J. Optimization, 6 (1996), 164--177.

  \bibitem{Rockafellar70}
  R.T. Rockafellar. Convex Analysis. Princeton University Press, Princeton, New Jersey, 1970.

  \bibitem{Rockafellar98}
  R.T. Rockafellar and R.J.B Wets. Variational Analysis. Springer-Verlag, New York, 1998.

  \bibitem{Seeger97}
  A. Seeger. Convex analysis of spectrally defined matrix functions. SIAM J. Optimization, 7 (1997), 679--696.

  \end{thebibliography}

\end{document}
