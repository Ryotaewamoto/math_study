\documentclass[a4paper,11pt, oneside]{book}
% 数式
\usepackage{amsmath,amsfonts, amsthm, amssymb}
\usepackage{bm}
\usepackage{mathtools}
% 画像
\usepackage[dvipdfmx]{graphicx}
% ダミーテキスト
\usepackage{blindtext}
% 箇条書き
\usepackage{enumitem}

\SetLabelAlign{Center}{\hfil#1\hfil}
\SetLabelAlign{CenterWithParen}{\hfil(\makebox[1.0em]{#1})\hfil}

\theoremstyle{definition}
\newtheorem{dfn}{Definition}[section]
\newtheorem{prop}[dfn]{Proposition}
\newtheorem{lem}[dfn]{Lemma}
\newtheorem{thm}[dfn]{Theorem}
\newtheorem{cor}[dfn]{Corollary}
\newtheorem{rem}[dfn]{Remark}
\newtheorem{fact}[dfn]{Fact}

% ヘッダーにチャプター名を表示しない。
\pagestyle{plain}

% 行間の設定
\renewcommand{\baselinestretch}{1.3}

% Mathematical Sets
\newcommand{\NaturalNumberSet}{\mathbb{N}}
\newcommand{\RealNumberSet}{\mathbb{R}}
\newcommand{\NDemenstionalRealEuclideanSpace}{\mathbb{R}^n}
\newcommand{\NDemenstionalRealSymmetricMatrixSpace}{\mathbb{S}^n}

% Symbols like prefix
\newcommand{\Closure}[1]{\text{\rm cl\:${#1}$}} % cl
\newcommand{\Interior}[1]{\text{\rm int\:${#1}$}} % int
\newcommand{\Domain}[1]{\text{\rm dom\:${#1}$}} % dom
\newcommand{\Epigraph}[1]{\text{\rm epi\:${#1}$}} % epi
\newcommand{\Trace}[1]{\text{\rm tr$({#1})$}} % tr
\newcommand{\Diagnosis}[1]{\text{\rm diag\:${#1}$}} % diag
\newcommand{\InnerProduct}[2]{\left\langle {#1},{#2}\right\rangle} % <x,y>

% Extended real valued function e.g. f: X -> Rv{+∞}
% #1: function symbol
% #2: domain of function
\newcommand{\ExtendedRealValuedFunction}[2]{{#1}: {#2} \to \RealNumberSet \cup \{+\infty\}}

% Conjugate function e.g. f*
% #1: function symbol
\newcommand{\ConjugateFunction}[1]{{#1}^*}

% Support function e.g. f*
% #1: set symbol
\newcommand{\SupportFunction}[1]{\sigma_{#1}}

% Indicator function e.g. f*
% #1: set symbol
\newcommand{\IndicatorFunction}[1]{\delta_{#1}}

% (Useful) Texts
\newcommand{\SuchThat}{\:\text{s.t.}\:}

% Set form e.g. {x | ...}
% #1: element
% #2: conditions
\newcommand{\SetForm}[2]{
  \{{#1}\:|\:{#2}\}
}

\begin{document}
\begin{titlepage}
\begin{center}
\vspace*{10mm}
{\bf \Huge Convex Analysis and Fundamental of Asymptotic Cones and Functions}
\vspace{80mm}

{\bf \Huge Ryota Iwamoto}\\
\vspace{15mm}
{\huge Master's Program in Fundamental Sciences}\\
\vspace{5mm}
{\huge Graduate School of Science and Technology}\\
\vspace{5mm}
{\huge Niigata University}\\
\vspace{15mm}
{\huge March 2024}
\end{center}
\end{titlepage}

\tableofcontents

% 1. Introduction
% ----------------------------------------------------------------
\chapter{Introduction}
This is the first section.
\blindtext

\blindtext


% 2. Preliminaries
% ----------------------------------------------------------------
\chapter{Preliminaries}

$\NDemenstionalRealEuclideanSpace$: $n$-dimensional real Euclidean space. \\
$\NDemenstionalRealSymmetricMatrixSpace$: $n$-dimensional real symmetric matrix space.

The inner product of $\NDemenstionalRealEuclideanSpace$ $\left\langle \cdot ,\cdot \right\rangle$  is defined by

  \begin{equation}
    \InnerProduct{x}{y} \coloneqq \sum_{i = 1}^{n} x_i y_i \:\text{for}\: x=(x_1,\dots,x_n)^T \in \mathbb{R}^n \:\text{and}\: y=(y_1,\dots,y_n)^T \in \mathbb{R}^n. \notag
  \end{equation}

The norm is defined by $\left\lVert x \right\rVert \coloneqq \InnerProduct{x}{x} ^{1/2} $. Like that, we can define the inner product and the norm of $\NDemenstionalRealSymmetricMatrixSpace$.

  \begin{equation}
    \InnerProduct{X}{Y} \coloneqq \Trace{XY} \:\text{for}\: X,Y \in \NDemenstionalRealSymmetricMatrixSpace \quad \text{and} \quad \left\lVert X \right\rVert \coloneqq \InnerProduct{X}{X} ^{1/2} \notag
  \end{equation}

% 3. Convex Sets and Functions
% ----------------------------------------------------------------
\chapter{Convex Sets and Functions}

Convexity is one of the most important concepts in convex analysis. In this chapter, we mainly introduce the definition of convex sets and functions.

\section{Convex Sets}

\begin{dfn}
  A subset $C$ of $\NDemenstionalRealEuclideanSpace$ is convex if
  \begin{equation}
    tx + (1-t)y \in C,\:\forall x,y \in C,\:t \in [0,1].\notag
  \end{equation}
\end{dfn}
If a given set is empty, for convenience, we define that the set is convex.

We recall some basic topological concepts of sets. The closed unit ball in $n$ dimensional Euclidean space $\NDemenstionalRealEuclideanSpace$ will be denoted by:

\begin{equation}
  \it{B} = \SetForm{x \in \NDemenstionalRealEuclideanSpace}{\left\lVert x \right\rVert \leq \text{1}}. \notag
\end{equation}

The ball considering the center $x_0$ and the radius $\delta$ is written as $\it{B}(x_0, \delta) \coloneqq x_0 + \delta \it{B}$. Let $C \subset \NDemenstionalRealEuclideanSpace$ be a nonempty set. The interior and closure of $C$ are defined respectively by $\text{int}\:C$ and $\text{cl}\:C$, respectively.

\begin{equation}
  \begin{split}
    \Interior{C} &\coloneqq \SetForm{x \in C}{\exists \epsilon > 0 \:\text{such that}\: x +\epsilon\it{B} \subset C}, \\
    \Closure{C} &\coloneqq \bigcup_{\epsilon > 0}{C + \epsilon\it{B}}. \notag \\
    \end{split}
\end{equation}

The boundary of $C$ is denoted by $\partial C$.

\begin{dfn}
  A subset $C \subset \NDemenstionalRealEuclideanSpace$ is called a cone if
  \begin{equation}
    tx \in C,\:\forall x \in C, t\geq0. \notag
  \end{equation}
\end{dfn}

\section{Convex Functions}
\begin{dfn}
  A function $f$ is sublinear if $f$ satisfies the condition

  \begin{equation}
    f(\lambda x + \mu y) \leq \lambda f(x) + \mu f(y),\forall x, y \in{\NDemenstionalRealEuclideanSpace},\: \lambda, \mu \in{\RealNumberSet}_+. \notag
  \end{equation}
\end{dfn}

\section{Conjugate Functions}

\begin{dfn}
  Let $\ExtendedRealValuedFunction{f}{\NDemenstionalRealEuclideanSpace}$. The conjugate function of $f$ is defined by
  \begin{equation}
    \ConjugateFunction{f}(y) \coloneqq \sup_{x \in \NDemenstionalRealEuclideanSpace} \{ \InnerProduct{x}{y} - f(x) \}. \notag
  \end{equation}
\end{dfn}

\begin{prop}{(Fenchel-Young Inequality)}
  Let $\ExtendedRealValuedFunction{f}{\NDemenstionalRealEuclideanSpace}$ be a proper function. Then
  \begin{equation}
    f(x) + \ConjugateFunction{f}(y) \geq \InnerProduct{x}{y}, \forall x,y \in \NDemenstionalRealEuclideanSpace. \notag
  \end{equation}
  and the equality holds if and only if $y \in \partial f(x)$.
\end{prop}

\begin{proof}
  TODO
\end{proof}

We now provide the definition of conjugate functions in $\NDemenstionalRealSymmetricMatrixSpace$.

\begin{dfn}
  Let $\ExtendedRealValuedFunction{\Phi}{\NDemenstionalRealSymmetricMatrixSpace}$ be a function. The conjugate function $\ConjugateFunction{\Phi}$ of $\Phi$ is defined by
  \begin{equation}
    {\Phi_{f}}^* (Y) \coloneqq \sup \{\InnerProduct{X}{Y} - \Phi_{f} (X) \:|\: X \in \NDemenstionalRealSymmetricMatrixSpace\}, \forall Y \in \NDemenstionalRealSymmetricMatrixSpace. \notag
  \end{equation}
\end{dfn}

\section{Support Functions}

\begin{dfn}
  Given a nonempty set $C \subset \NDemenstionalRealEuclideanSpace$, the function $\ExtendedRealValuedFunction{\sigma_C}{\NDemenstionalRealEuclideanSpace}$ defined by
  \begin{equation}
    \sigma_C (d) \coloneqq \sup\{\InnerProduct{x}{d} \:|\: x \in C\} \notag
  \end{equation}
  is called the support function of $C$.
\end{dfn}

% 4. Asymptotic Cones
% ----------------------------------------------------------------
\chapter{Asymptotic Cones and Functions}

\section{Definitions of Asymptotic Cones}

\begin{dfn}
  p.26 definition 2.1.1
\end{dfn}

\begin{dfn}
  $C \subset \mathbb{R}^n$, $C \neq \emptyset$. Then, the asymptotic cone of the set $C$, denoted by $C_\infty$, is the set below with $\{ x_k \} \subset C$;
  \begin{equation}
    C_\infty = \left\{ d \in
    \mathbb{R}^n \:\middle|\: \exists t_k \rightarrow +\infty, \exists x_k \in C \:\text{\rm with}\: \lim_{k \to \infty} \frac{x_k}{t_k} = d \right\}. \notag
  \end{equation}
\end{dfn}

\begin{prop}\label{basicPropositionOfAsymptoticCone}
  p.50 proposition2.1.1
\end{prop}

\begin{prop}
  A set $C \subset \mathbb{R}^n$ is bounded if and only if $C_\infty = \{0\}$.
\end{prop}

\begin{dfn}
  Let $C \subset \NDemenstionalRealEuclideanSpace$ be nonempty and define
  \begin{equation}
    C_{\infty}^1 = \left\{d \in \NDemenstionalRealEuclideanSpace \:\middle|\: \forall t_k \rightarrow + \infty , \exists x_k \in C \:\text{ with }\: \lim_{k \to \infty} \frac{x_k}{t_k} = d \right\}. \notag
  \end{equation}
  We say that $C$ is asymptotically regular if $C_{\infty} = C_{\infty}^1$.
\end{dfn}

\begin{prop}
  Let $C$ be a nonempty convex set in $\NDemenstionalRealEuclideanSpace$. Then $C$ is asymptotically regular.
\end{prop}

\begin{prop}
  p.27 proposition2.1.4
\end{prop}

\begin{prop}
  p.27 proposition2.1.5
\end{prop}

\begin{prop}
  p.30 proposition2.1.9
\end{prop}

\begin{prop}
  p.31 proposition2.1.10
\end{prop}

\section{Asymptotic Functions}

\begin{dfn}
  For any proper function $\ExtendedRealValuedFunction{f}{\NDemenstionalRealEuclideanSpace}$, there exists a unique function $\ExtendedRealValuedFunction{f_{\infty}}{\NDemenstionalRealEuclideanSpace}$ associated with $f$, called the asymptotic function, such that $\Epigraph{f_\infty} = (\Epigraph{f})_{\infty}$.
\end{dfn}

\begin{prop}
  For any proper function $\ExtendedRealValuedFunction{f}{\NDemenstionalRealEuclideanSpace}$, we have:
  \begin{enumerate}[label=\roman*,align=CenterWithParen]
    \item $f_{\infty}$ is lsc and positively homogeneous.
    \item $f_{\infty}(0) = 0$ or $f_{\infty}(0) = - \infty$.
    \item If $f_{\infty}(0) = 0$, then $f_{\infty}$ is proper.
  \end{enumerate}
\end{prop}

\begin{proof}
  We show each statement.
  \begin{enumerate}[label=\roman*,align=CenterWithParen]
    \item By the definition of the asymptotic function and the fact that asymptotic cones are closed (see Proposition \ref{basicPropositionOfAsymptoticCone}), it follows that $f_{\infty}$ is l.s.c. \\
    First, note that $0 \in \Domain{f_{\infty}}$. Let $x \in \Domain{f_{\infty}}$. Since $(\Epigraph{f})_{\infty}$ is a cone, we have
    \begin{equation}
      (\lambda x, \lambda f_{\infty}(x)) \in \Epigraph{f_{\infty}}, \quad \forall \lambda > 0. \notag
    \end{equation}
    By the definition of epigraph, we obtain $f_{\infty} (\lambda x) \leq \lambda f_{\infty}(x)$. \\
    Likewise, $\forall x \in \Domain{f_{\infty}}, \lambda > 0$, $(\lambda x, f_{\infty}(\lambda x)) \in \Epigraph{f_{\infty}}$ and hence one has $(x, \lambda^{-1}f_{\infty}(\lambda x)) \in \Epigraph{f_\infty}$ by the cone property. Therefore, $\lambda f_{\infty}(x) \leq f_{\infty} (\lambda x)$
    \item Since $f$ is proper, then $\Epigraph{f}$ is nonempty, and hence either $f_{\infty}(0)$ is finite or $f_{\infty}(0) = - \infty$. If $f_{\infty}(0)$ is finite, then by the result of (i), $f_{\infty}(0) = \lambda f_{\infty}(0), \forall \lambda > 0$. Thus, $f_{\infty}(0) = 0$.
  \item Suppose that $f_{\infty}$ is not proper. Then, by the definition, there exists $x$ such that $f_{\infty}(x) = -\infty$. Now let $\{\lambda_k\} \subset \NDemenstionalRealEuclideanSpace_{++}$ be a positive sequence converging to $0$. Then $\lambda_k x \rightarrow 0$, and under our assumption, the lower semicontinuity of $f_{\infty}$, and the result of (i), we have
  \begin{equation}
    0 = f_{\infty}(0) \leq \liminf_{k \to \infty} f_{\infty}(\lambda_k x) = \liminf_{k \to \infty} \lambda_k f_{\infty}(x) = - \infty, \notag
  \end{equation}
  leading to a contradiction. Therefore, $f_{\infty}$ is proper.
  \end{enumerate}
\end{proof}

We now give a fundamental representation of the asymptotic function.

\begin{thm}
  For any proper function $\ExtendedRealValuedFunction{f}{\NDemenstionalRealEuclideanSpace}$, the asymptotic function $f_{\infty}$ is represented as
  \begin{equation}
    f_{\infty}(d) = \liminf_{t \to +\infty, d' \to d} \frac{f(td')}{t},
  \end{equation}
  or equivalently,
  \begin{equation}
    f_{\infty}(d) = \inf \left\{ \liminf_{k \to \infty} \frac{f(t_k d_k)}{t_k} \:|\: t_k \to +\infty, d_k \to d \right\},
  \end{equation}
  where $\{t_k\}$ and $\{d_k\}$ are sequences in $\RealNumberSet$ and  $\NDemenstionalRealEuclideanSpace$, respectively.
\end{thm}

\begin{proof}
  TODO
\end{proof}

\begin{prop}
  Let $\ExtendedRealValuedFunction{f}{\NDemenstionalRealEuclideanSpace}$ be a proper, l.s.c, convex function. The asymptotic function $f_{\infty}$ is positively homogeneous, l.s.c., proper convex function, and for any $d \in \NDemenstionalRealEuclideanSpace$, one has
  \begin{equation}
    f_{\infty}(d) = \sup \{f(x+d) -f(x) \:|\: x \in \Domain{f}\}
  \end{equation}
  and
  \begin{equation}
    f_{\infty}(d) = \lim_{t \to +\infty} \frac{f(x+td) -f(x)}{t} = \sup_{t>0} \frac{f(x+td) - f(x)}{t}, \forall x \in \Domain{f}.
  \end{equation}
\end{prop}

\begin{proof}
  TODO
\end{proof}

\begin{thm}
  Let $\ExtendedRealValuedFunction{f}{\NDemenstionalRealEuclideanSpace}$ be a proper convex function, and $\ConjugateFunction{f}$ its conjugate. The following relations hold:
  \begin{enumerate}[label=\alph*,align=CenterWithParen]
    \item $(\ConjugateFunction{f})_{\infty} = \SupportFunction{\Domain{f}}$.
    \item If $f$ is also assumed l.s.c., then
    \begin{equation}
      f_{\infty} = \SupportFunction{\Domain{\ConjugateFunction{f}}},\quad \ConjugateFunction{f_{\infty}} = \IndicatorFunction{\Closure{\Domain{\ConjugateFunction{f}}}}. \notag
    \end{equation}
  \end{enumerate}
\end{thm}

\begin{proof}
  TODO
\end{proof}

% 5. Application of Asymptotic Functions to Semidefinite Programming
% ----------------------------------------------------------------
\chapter{Application of Asymptotic Functions to Semidefinite Programming}

This chapter can be considered as the main part of the paper. Semidefinite programming (SDP) is one of the most significant optimization problems because these problems have appeared in several areas of mathematical sciences and engineering; and have been expected as a wide rage of application. At first, we introduce the definition of SDP and some examples of SDP. After that, we show relations between SDP and asymptotic functions.

\section{Semidefinite Programming}

Mathematical optimization problems are initially classified into two types, continuous optimization problem and discrete optimization problem. In continuous optimization problem, the variables of the problem are continuos. On the other hand, in the case that the variables can only be taken as discrete values like 0 or 1, the problem is called discrete optimization. General mathematical programming have twe types, convex programming and non-convex programming. Convex programming is a problem of minimizing a convex function subject to convex constraints of the domain. SDP we centrally consider is one of the convex programming problems and has several famous mathematical optimization problems like liner programming, convex quadratic programming, and second-order cone programming.

When we suppose that $C \in \NDemenstionalRealSymmetricMatrixSpace$, $A_i \in \NDemenstionalRealSymmetricMatrixSpace$, $b_i \in \RealNumberSet$, $i = 1,\dots,m$, and $X \in \NDemenstionalRealSymmetricMatrixSpace$, which is a variable on the problem, the following constrained problem is called semidefinite programming (SDP).

\begin{equation}
  \begin{aligned}
  \min \quad & \InnerProduct{C}{X}\\
  \textrm{s.t.} \quad & \InnerProduct{A_i}{X} = b_i, \quad i = 1,\dots,m\\
    & X \succcurlyeq 0. \\ \notag
  \end{aligned}
\end{equation}
\section{Spectrally Defined Matrix Functions}

Spectrally defined matrix functions are often seen in Semidefinite programming. They have strong connection with matrix optimizations, called semidefinite programming. At first, we introduce a key concept of matrix optimization, symmetric functions.

\begin{dfn}
  Let $\ExtendedRealValuedFunction{f}{\NDemenstionalRealEuclideanSpace}$.
  $f$ is said to be symmetric if
  \begin{equation}
    \forall x \in \NDemenstionalRealEuclideanSpace \:\text{and}\: P: n \times n \text{ permutation matrix}, f(Px) = f(x). \notag
  \end{equation}
\end{dfn}
For example, each of the following functions is symmetric.
\begin{enumerate}[label=\roman*,align=CenterWithParen]
  \item $f(x) = \max_{1 \leq i \leq n} x_i$ (or $\min_{1 \leq i \leq n} x_i$),
  \item $f(x) = \sum_{i = 1}^{n} x_i$ (or $\prod_{i = 1}^{n} x_i$).
\end{enumerate}

\begin{dfn}
  The function $\ExtendedRealValuedFunction{\Phi}{\NDemenstionalRealSymmetricMatrixSpace}$ is said to be spectrally defined if there exists a symmetric function $\ExtendedRealValuedFunction{f}{\NDemenstionalRealEuclideanSpace}$ such that
  \begin{equation}
    \Phi (X) = \Phi_{f}(X) \coloneqq f(\lambda (X)), \forall X \in \NDemenstionalRealSymmetricMatrixSpace \notag
  \end{equation}
  where $\lambda (X) \coloneqq (\lambda_1 (X), \dotsb , \lambda_n (X))^T$ is the vector of eigenvalues of $X$ in nondecreasing order.
\end{dfn}

When we define the following symmetric function
\begin{equation}
  f(\lambda)=
    \begin{cases}
      - \Sigma_{i = 1}^{n} \log \lambda_i & \text{\rm if}\:\lambda > 0; \\
      +\infty & \text{\rm otherwise}, \notag
    \end{cases}
\end{equation}
the spectrally defined function is deduced;
\begin{equation}
  \Phi_{f}(X)=
    \begin{cases}
      - \log \det (X) & \text{\rm if}\:X \succ 0; \\
      +\infty & \text{\rm otherwise}. \notag
    \end{cases}
\end{equation}

\begin{prop}
  The $\Phi$ is spectrally defined if and only if $\Phi$ is othonormally in variant, that is,
  \begin{equation}
    \Phi (UAU^T) = \Phi (A), \forall A \in \mathcal{U}_n, \notag
  \end{equation}
  where $\mathcal{U}_n \coloneqq$ the set $n \times n$ orthogonal matrices.
\end{prop}

\begin{proof}
  TODO
\end{proof}

To prove the following key theorem, we introduce two properties of symmetric matrices.

\begin{lem}\label{lemma1ForLewis96}
  $\forall Y \in \NDemenstionalRealSymmetricMatrixSpace$, there exists a orthogonal matrix $U_0$ diagonalizing $Y$ and satisfying
  \begin{equation}
    \forall X \in \NDemenstionalRealSymmetricMatrixSpace, \InnerProduct{X}{Y} = \InnerProduct{U_0^TXU_0}{\Diagnosis{\lambda (Y)}}. \notag
  \end{equation}
\end{lem}

\begin{proof}
  Since Y is an element in $\NDemenstionalRealSymmetricMatrixSpace$, there exists a orthogonal matrix $U_0$ diagonalizing $Y$. Then, for all $X \in \NDemenstionalRealSymmetricMatrixSpace$, we have
  \begin{equation}
    \begin{split}
      \InnerProduct{X}{Y} &= \Trace{XY} = \Trace{XYU_0U_0^T} = \Trace{U_0^TXYU_0} \\
      &= \Trace{U_0^TXU_0U_0^TYU_0}\\
      &= \InnerProduct{U_0^TXU_0}{U_0^TYU_0}\\
      &= \InnerProduct{U_0^TXU_0}{\Diagnosis{\lambda (Y)}}. \notag
    \end{split}
  \end{equation}
\end{proof}

\begin{lem}\label{lemma2ForLewis96}
  Let $B: \NDemenstionalRealSymmetricMatrixSpace \to \NDemenstionalRealSymmetricMatrixSpace$. Suppose to $B(X) = U^TXU$ with $U \in \mathcal{U}_n$ where $\mathcal{U}_n \coloneqq$ the set $n \times n$ orthogonal matrices. Then $U^TXU \in \NDemenstionalRealSymmetricMatrixSpace, \forall X \in \NDemenstionalRealSymmetricMatrixSpace$ and $B$ is surjective.
\end{lem}

\begin{proof}
  $\forall X \in \NDemenstionalRealSymmetricMatrixSpace, U \in \mathcal{U}_n$, since $X$ is symmetric,
  \begin{equation}
    (U^TXU)^T = U^TX^TU = U^TXU. \notag
  \end{equation}
  Thus, $U^TXU$ is an element of $\NDemenstionalRealSymmetricMatrixSpace$. \\
  In addition, we show that  $\forall Y \in \NDemenstionalRealSymmetricMatrixSpace, \exists X \in \NDemenstionalRealSymmetricMatrixSpace \SuchThat B(X) = Y$. For any $Y \in \NDemenstionalRealSymmetricMatrixSpace$, we put $X = UYU^T$. Then, we obtain $B(X) = U^TXU = Y$. Therefore, $B$ is surjective.
\end{proof}

\begin{thm}{(A.S.Lewis (1996), \cite{Lewis96})}
  Suppose that the function $\ExtendedRealValuedFunction{f}{\NDemenstionalRealEuclideanSpace}$ is symmetric, then
  \begin{equation}
    {\Phi_{f}}^* = \Phi_{f^*}. \notag
  \end{equation}
\end{thm}

\begin{proof}
  By the results of Lemma \ref{lemma1ForLewis96} and Lemma \ref{lemma2ForLewis96}, we have $\ConjugateFunction{\Phi_f}(Y) = \ConjugateFunction{\Phi}(\Diagnosis{\lambda (Y)})$. Using the definition of the conjugate functions, one thus has
  \begin{equation}
    \begin{split}
      \ConjugateFunction{\Phi_f}(Y) &= \ConjugateFunction{\Phi}(\Diagnosis{\lambda (Y)})\\
      &= \sup_{X \in \NDemenstionalRealSymmetricMatrixSpace} \{\InnerProduct{X}{\Diagnosis{\lambda (Y)}} - \Phi_f(X)\}\\
      &\geq \sup_{x \in \NDemenstionalRealEuclideanSpace} \{\InnerProduct{\Diagnosis{x}}{\Diagnosis{\lambda (Y)}} - \Phi_f(\Diagnosis{x})\}\\
      &= \sup_{x \in \NDemenstionalRealEuclideanSpace} \{\InnerProduct{x}{\lambda (Y)} - f(x)\}\\
      &= \ConjugateFunction{f}(\lambda (Y)) = \Phi_{\ConjugateFunction{f}}(Y). \notag
    \end{split}
  \end{equation}
  The proof of the reverse inequality $\ConjugateFunction{\Phi_f}(Y) \leq \Phi_{\ConjugateFunction{f}}(Y)$ is obtained by combining the Fenchel-Young inequality $\ConjugateFunction{f}(\lambda(Y)) + f(\lambda(X)) \geq \InnerProduct{\lambda(X)}{\lambda(Y)}$ with the trace inequality. Moving to the details of the proof, it holds that $\forall X \in \NDemenstionalRealSymmetricMatrixSpace$,
  \begin{equation}
    \begin{split}
      \Phi_{\ConjugateFunction{f}}(Y) &\geq \InnerProduct{\lambda(X)}{\lambda(Y)} - f(\lambda(X))\\
      &\geq \InnerProduct{X}{Y} - f(\lambda(X))\\
      &= \InnerProduct{X}{Y} - \Phi_f(X). \notag
    \end{split}
  \end{equation}
  As a result, we obtain $\ConjugateFunction{\Phi_f}(Y) \geq \Phi_{\ConjugateFunction{f}}(Y) = \ConjugateFunction{\Phi_f}(Y)$.
\end{proof}

\begin{prop}
  Let $\ExtendedRealValuedFunction{f}{\NDemenstionalRealEuclideanSpace}$ be a symmetric proper convex and lsc function, $\ExtendedRealValuedFunction{\Phi_f}{\NDemenstionalRealSymmetricMatrixSpace}$ the induced spectral function. Then, $\ConjugateFunction{f}$ is symmetric, and the following relations hold:
  \begin{enumerate}[label=\roman*,align=CenterWithParen]
    \item $\Phi_{f}$ is proper, convex, and lsc.
    \item $\ConjugateFunction{\Phi_f}(Y) + \Phi_f(X) \geq \InnerProduct{\lambda(X)}{\lambda(Y)}, \forall X,Y \in \NDemenstionalRealSymmetricMatrixSpace$.
    \item $Y \in \partial \Phi_f(X) \Leftrightarrow \InnerProduct{\lambda(X)}{\lambda(Y)} = \InnerProduct{X}{Y} and \lambda(Y) \in \partial f(\lambda(x))$.
  \end{enumerate}
\end{prop}

\begin{proof}
  We now prove each statement.
  \begin{enumerate}[label=\roman*,align=CenterWithParen]
    \item It is obvious by the definition of the induced spectral function $\Phi_f$ and the assumption of $f$.
    \item By the definition of the conjugate function, it holds that $\forall Y \in \NDemenstionalRealSymmetricMatrixSpace$,
    \begin{equation}
      \begin{split}
        \ConjugateFunction{\Phi}(Y) &= \sup_{X \in \NDemenstionalRealSymmetricMatrixSpace} \{\InnerProduct{X}{Y} - \Phi(X)\} \\
        &\geq \InnerProduct{X}{Y} - \Phi(X), \quad \forall X \in \NDemenstionalRealSymmetricMatrixSpace. \notag
      \end{split}
    \end{equation}
    Thus, we obtain $\ConjugateFunction{\Phi}(Y) + \Phi(X) \geq \InnerProduct{X}{Y}$.
    \item We show ($\Rightarrow$) and ($\Leftarrow$).

    ($\Rightarrow$) By Fenchel-Young inequality, the definition of the spectral function, the assumption, the result of (ii) and the trace inequality, we have
    \begin{equation}
      \begin{split}
        \InnerProduct{\lambda(X)}{\lambda(Y)} &\leq \ConjugateFunction{f}(\lambda(Y)) + f(\lambda(X)) \\
        &= \ConjugateFunction{\Phi}(Y) + \Phi(X) \\
        &= \InnerProduct{X}{Y} \\
        &\leq \InnerProduct{\lambda(X)}{\lambda(Y)}. \notag
      \end{split}
    \end{equation}
    Additionally, for $X$ and $Y$ satisfying $Y \in \partial \Phi_f(X)$, one has
    \begin{equation}
      \begin{split}
        \Phi(Z) &\geq \Phi(X) + \InnerProduct{Z-X}{Y}, \quad \forall Z \in \NDemenstionalRealSymmetricMatrixSpace \\
        f(\lambda(Z)) &\geq f(\lambda(X)) + \InnerProduct{\lambda(Z)}{\lambda(Y)} - \InnerProduct{\lambda(X)}{\lambda(Y)}, \quad \forall Z \in \NDemenstionalRealSymmetricMatrixSpace\\
        f(z) &\geq f(\lambda(X)) + \InnerProduct{z - \lambda(X)}{\lambda(Y)}, \quad \forall z \in \NDemenstionalRealEuclideanSpace. \notag
      \end{split}
    \end{equation}
    Thus, $\lambda(Y) \in \partial f(\lambda(X))$.

    ($\Leftarrow$) It is clear by the proof of ($\Rightarrow$).
  \end{enumerate}
\end{proof}

As a result, the optimization problems $\min \{\Phi_{f}(X) \:|\: X \in \NDemenstionalRealSymmetricMatrixSpace\}$ and $\min \{f(x) \:|\: x \in \NDemenstionalRealEuclideanSpace\}$ are equivalent. In fact,

\begin{align}
  \inf_{X \in \NDemenstionalRealSymmetricMatrixSpace} \Phi_{f}(X) &= - \sup_{X \in \NDemenstionalRealSymmetricMatrixSpace} \{- \Phi_{f}(X)\} = - \sup_{X \in \NDemenstionalRealSymmetricMatrixSpace} \{\InnerProduct{X}{0} - \Phi_{f}(X)\} \notag \\
  &= - \Phi^*_{f}(0) = - \Phi_{\ConjugateFunction{f}}(0) = - \ConjugateFunction{f}(0) = \inf_{x \in \NDemenstionalRealEuclideanSpace} f(x). \notag
\end{align}

\begin{dfn}
  The asymptotic functions of the proper convex lsc function $\ExtendedRealValuedFunction{\Phi}{\NDemenstionalRealSymmetricMatrixSpace}$ is defined by, for all $D \in \NDemenstionalRealSymmetricMatrixSpace$
  \begin{equation}
    \begin{split}
      \Phi_{\infty} (D) &= \sup_{t > 0} \frac{\Phi(A+tD) -\Phi(A)}{t}, \forall A \in \Domain{\Phi} \quad \text{and} \\
      &= \sup \{\InnerProduct{B}{D} \:|\: B \in \Domain{\ConjugateFunction{\Phi}}\}. \notag
    \end{split}
  \end{equation}
\end{dfn}

\begin{thm}{(A.Seeger (1997), \cite{Seeger97})}
  Let $\ExtendedRealValuedFunction{f}{\NDemenstionalRealEuclideanSpace}$ be a symmetric, lsc, proper, convex function with induced spectral function $\Phi_{f}$. Then
  \begin{equation}
    {(\Phi_{f})}_{\infty} = \Phi_{f_{\infty}}. \notag
  \end{equation}
\end{thm}

\begin{proof}
  TODO
\end{proof}
% 6. Conclusion
% ----------------------------------------------------------------
\chapter{Conclusion}
\blindtext

\begin{thebibliography}{9}
  \bibitem{Auslender03}
  A. Auslender and M. Teboulle, Asymptotic cones and functions in optimization and variational inequalities, Springer monographs in Mathematics, Springer-Verlag, New York, 2003.

  \bibitem{Auslender03}
  J.M. Borwein and A.S. Lewis, Convex Analysis and Nonlinear Optimization: Theory and Examples, Springer-Verlag, New York, 2000.

  \bibitem{Tammer03}
  A. G\"{o}pfert, H. Riahi, C. Tammer, and C. Z\u{a}linescu, Variational methods in partially ordered spaces, vol. 17 of CMS Books in Mathematics, Springer-Verlag, New York, 2003.

  \bibitem{Lewis96}
  A.S. Lewis. Convex Analysis on the Hermitian matrices. SIAM J. Optimization,6, 1996, 164-177.

  \bibitem{Seeger97}
  A. Seeger. Convex analysis of spectrally defined matrix functions. SIAM J. Optimization , 7, 1997, 679-696.

  \end{thebibliography}
\end{document}

\end{document}
