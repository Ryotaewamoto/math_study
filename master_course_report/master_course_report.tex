\documentclass[a4paper,11pt, oneside]{book}
% 数式
\usepackage{amsmath,amsfonts, amsthm, amssymb}
\usepackage{bm}
\usepackage{mathtools}
% 画像
\usepackage[dvipdfmx]{graphicx}
% 箇条書き
\usepackage{enumitem}


\SetLabelAlign{Center}{\hfil#1\hfil}
\SetLabelAlign{CenterWithParen}{\hfil(\makebox[1.0em]{#1})\hfil}
\SetLabelAlign{CenterWithParen2}{\hfil(\makebox[1.5em]{#1})\hfil}

\theoremstyle{definition}
\newtheorem{dfn}{Definition}[section]
\newtheorem{prop}[dfn]{Proposition}
\newtheorem{lem}[dfn]{Lemma}
\newtheorem{thm}[dfn]{Theorem}
\newtheorem{cor}[dfn]{Corollary}
\newtheorem{rem}[dfn]{Remark}
\newtheorem{fact}[dfn]{Fact}

% ヘッダーにチャプター名を表示しない。
\pagestyle{plain}

% 行間の設定
\renewcommand{\baselinestretch}{1.3}

% Mathematical Sets
\newcommand{\NaturalNumberSet}{\mathbb{N}}
\newcommand{\RealNumberSet}{\mathbb{R}}
\newcommand{\NDemenstionalRealEuclideanSpace}{\mathbb{R}^n}
\newcommand{\NDemenstionalRealSymmetricMatrixSpace}{\mathbb{S}^n}

% Symbols like prefix
\newcommand{\Closure}[1]{\text{\rm cl\:${#1}$}} % cl
\newcommand{\Interior}[1]{\text{\rm int\:${#1}$}} % int
\newcommand{\Domain}[1]{\text{\rm dom\:${#1}$}} % dom
\newcommand{\Epigraph}[1]{\text{\rm epi\:${#1}$}} % epi
\newcommand{\Trace}[1]{\text{\rm tr$({#1})$}} % tr
\newcommand{\Diagnosis}[1]{\text{\rm diag\:${#1}$}} % diag
\newcommand{\InnerProduct}[2]{\left\langle {#1},{#2}\right\rangle} % <x,y>

% Extended real valued function e.g. f: X -> Rv{+∞}
% #1: function symbol
% #2: domain of function
\newcommand{\ExtendedRealValuedFunction}[2]{{#1}: {#2} \to \RealNumberSet \cup \{+\infty\}}

% Conjugate function e.g. f*
% #1: function symbol
\newcommand{\ConjugateFunction}[1]{{#1}^*}

% Support function e.g. f*
% #1: set symbol
\newcommand{\SupportFunction}[1]{\sigma_{#1}}

% Indicator function e.g. f*
% #1: set symbol
\newcommand{\IndicatorFunction}[1]{\delta_{#1}}

% (Useful) Texts
\newcommand{\SuchThat}{\:\text{s.t.}\:}

% Set form e.g. {x | ...}
% #1: element
% #2: conditions
\newcommand{\SetForm}[2]{
  \{{#1}\:|\:{#2}\}
}

\begin{document}
\begin{titlepage}
\begin{center}
\vspace*{10mm}
{\bf \Huge Convex Analysis and Fundamental of Asymptotic Cones and Functions}
\vspace{80mm}

{\bf \Huge Ryota Iwamoto}\\
\vspace{15mm}
{\huge Master's Program in Fundamental Sciences}\\
\vspace{5mm}
{\huge Graduate School of Science and Technology}\\
\vspace{5mm}
{\huge Niigata University}\\
\vspace{15mm}
{\huge March 2024}
\end{center}
\end{titlepage}

\tableofcontents

% 1. Introduction
% ----------------------------------------------------------------
\chapter{Introduction}


% 2. Preliminaries
% ----------------------------------------------------------------
\chapter{Preliminaries}

$\NDemenstionalRealEuclideanSpace$: $n$-dimensional real Euclidean space. \\
$\NDemenstionalRealSymmetricMatrixSpace$: $n$-dimensional real symmetric matrix space.

The inner product of $\NDemenstionalRealEuclideanSpace$ $\left\langle \cdot ,\cdot \right\rangle$  is defined by

  \begin{equation}
    \InnerProduct{x}{y} \coloneqq \sum_{i = 1}^{n} x_i y_i \:\text{for}\: x=(x_1,\dots,x_n)^T \in \mathbb{R}^n \:\text{and}\: y=(y_1,\dots,y_n)^T \in \mathbb{R}^n. \notag
  \end{equation}

The norm is defined by $\left\lVert x \right\rVert \coloneqq \InnerProduct{x}{x} ^{1/2} $. Like that, we can define the inner product and the norm of $\NDemenstionalRealSymmetricMatrixSpace$.

  \begin{equation}
    \InnerProduct{X}{Y} \coloneqq \Trace{XY} \:\text{for}\: X,Y \in \NDemenstionalRealSymmetricMatrixSpace \quad \text{and} \quad \left\lVert X \right\rVert \coloneqq \InnerProduct{X}{X} ^{1/2} \notag
  \end{equation}

% 3. Convex Sets and Functions
% ----------------------------------------------------------------
\chapter{Convex Sets and Functions}

Convexity is one of the most important concepts in convex analysis. In this chapter, we mainly introduce the definition of convex sets and functions.

\section{Convex Sets}

\begin{dfn}
  A subset $C$ of $\NDemenstionalRealEuclideanSpace$ is convex if
  \begin{equation}
    tx + (1-t)y \in C,\:\forall x,y \in C,\:t \in [0,1].\notag
  \end{equation}
\end{dfn}
If a given set is empty, for convenience, we define that the set is convex.

We recall some basic topological concepts of sets. The closed unit ball in $n$ dimensional Euclidean space $\NDemenstionalRealEuclideanSpace$ will be denoted by:

\begin{equation}
  \it{B} = \SetForm{x \in \NDemenstionalRealEuclideanSpace}{\left\lVert x \right\rVert \leq \text{1}}. \notag
\end{equation}

The ball considering the center $x_0$ and the radius $\delta$ is written as $\it{B}(x_0, \delta) \coloneqq x_0 + \delta \it{B}$. Let $C \subset \NDemenstionalRealEuclideanSpace$ be a nonempty set. The interior and closure of $C$ are defined respectively by $\text{int}\:C$ and $\text{cl}\:C$, respectively.

\begin{equation}
  \begin{split}
    \Interior{C} &\coloneqq \SetForm{x \in C}{\exists \epsilon > 0 \:\text{such that}\: x +\epsilon\it{B} \subset C}, \\
    \Closure{C} &\coloneqq \bigcup_{\epsilon > 0}{C + \epsilon\it{B}}. \notag \\
    \end{split}
\end{equation}

The boundary of $C$ is denoted by $\partial C$.

\begin{dfn}
  A subset $C \subset \NDemenstionalRealEuclideanSpace$ is called a cone if
  \begin{equation}
    tx \in C,\:\forall x \in C, t\geq0. \notag
  \end{equation}
\end{dfn}

\section{Convex Functions}
\begin{dfn}
  A function $f$ is sublinear if $f$ satisfies the condition

  \begin{equation}
    f(\lambda x + \mu y) \leq \lambda f(x) + \mu f(y),\forall x, y \in{\NDemenstionalRealEuclideanSpace},\: \lambda, \mu \in{\RealNumberSet}_+. \notag
  \end{equation}
\end{dfn}

\section{Conjugate Functions}

\begin{dfn}
  Let $\ExtendedRealValuedFunction{f}{\NDemenstionalRealEuclideanSpace}$. The conjugate function of $f$ is defined by
  \begin{equation}
    \ConjugateFunction{f}(y) \coloneqq \sup_{x \in \NDemenstionalRealEuclideanSpace} \{ \InnerProduct{x}{y} - f(x) \}. \notag
  \end{equation}
\end{dfn}

\begin{prop}{(Fenchel-Young Inequality)}
  Let $\ExtendedRealValuedFunction{f}{\NDemenstionalRealEuclideanSpace}$ be a proper function. Then
  \begin{equation}
    f(x) + \ConjugateFunction{f}(y) \geq \InnerProduct{x}{y}, \forall x,y \in \NDemenstionalRealEuclideanSpace. \notag
  \end{equation}
  and the equality holds if and only if $y \in \partial f(x)$.
\end{prop}

\begin{proof}
  TODO
\end{proof}

We now provide the definition of conjugate functions in $\NDemenstionalRealSymmetricMatrixSpace$.

\begin{dfn}
  Let $\ExtendedRealValuedFunction{\Phi}{\NDemenstionalRealSymmetricMatrixSpace}$ be a function. The conjugate function $\ConjugateFunction{\Phi}$ of $\Phi$ is defined by
  \begin{equation}
    {\Phi_{f}}^* (Y) \coloneqq \sup \{\InnerProduct{X}{Y} - \Phi_{f} (X) \:|\: X \in \NDemenstionalRealSymmetricMatrixSpace\}, \forall Y \in \NDemenstionalRealSymmetricMatrixSpace. \notag
  \end{equation}
\end{dfn}

\section{Support Functions}

\begin{dfn}
  Given a nonempty set $C \subset \NDemenstionalRealEuclideanSpace$, the function $\ExtendedRealValuedFunction{\sigma_C}{\NDemenstionalRealEuclideanSpace}$ defined by
  \begin{equation}
    \sigma_C (d) \coloneqq \sup\{\InnerProduct{x}{d} \:|\: x \in C\} \notag
  \end{equation}
  is called the support function of $C$.
\end{dfn}

% 4. Asymptotic Cones
% ----------------------------------------------------------------
\chapter{Asymptotic Cones and Functions}

Chapter 4 provides the definition of asymptotic cones and functions. In addition to these definitions, we show some basic properties of asymptotic cones and functions.

\section{Definitions of Asymptotic Cones}

Before introducing the definition of asymptotic cones, we explain the motivation of considering asymptotic cones briefly. In classical analysis, in most cases, the boundedness of a given set is supposed naturally since the assumption leads to the existence of a limit of a sequence in the set by using Bolzano-Weierstrass theorem. However, in convex analysis, we are interested in existence of unbounded sets. Actually, the set of all feasible solutions of a convex optimization problem is unbounded in general. In such a case, we need to consider the limit of a sequence in the set. Asymptotic cones are tools to consider the limit of a sequence in an unbounded set.
The following statement is the definition of asymptotic cones.

\begin{dfn}
  $C \subset \mathbb{R}^n$, $C \neq \emptyset$. Then, the asymptotic cone of the set $C$, denoted by $C_\infty$, is the set below with $\{ x_k \} \subset C$;
  \begin{equation}
    C_\infty = \left\{ d \in
    \mathbb{R}^n \:\middle|\: \exists t_k \rightarrow +\infty, \exists x_k \in C \:\text{\rm with}\: \lim_{k \to \infty} \frac{x_k}{t_k} = d \right\}. \notag
  \end{equation}
\end{dfn}

\begin{prop}\label{basicPropositionOfAsymptoticCone}
  Let $C \subset \NDemenstionalRealEuclideanSpace$ be nonempty. Then:
  \begin{enumerate}[label=\roman*,align=CenterWithParen]
    \item $C_{\infty}$ is a closed cone.
    \item $(\text{cl}\:C)_{\infty} = C_{\infty}$.
    \item If $C$ is a cone, then $C_{\infty} = \text{cl}\:C$.
  \end{enumerate}
\end{prop}

\begin{proof}
  We will prove each part separately.
  \begin{enumerate}[label=\roman*,align=CenterWithParen]
    \item $C_{\infty}$ is a closed cone.
      We need to show two propositions: (i-a) $C_{\infty}$ is a cone and (i-b) $C_{\infty}$ is a closed set.
      \begin{enumerate}[label=i-\alph*,align=CenterWithParen]
        \item We show that $C_{\infty}$ is a cone, that is, $\forall \alpha \geq 0, d \in C_{\infty}, \alpha d \in C_{\infty}$. Since $0$ is an element of $C_{\infty}$, it is clear in the case of $\alpha = 0$.
        ($\because$ Since $C$ is nonempty, we can take an element $x_0$ from $C$. In addition we take a sequence $\{t_k\}_{k=1}^{\infty}$ with $t_k \rightarrow + \infty$ as $k \rightarrow \infty$. Obviously this sequence exists, for example $t_k \coloneqq k$. By using $t_k \coloneqq k$ and $x_k \coloneqq x_0$, we can obtain $0$ as the limit. Hence $0$ is an element of $C_{\infty}$.)
        Additionally, we consider the other case $\alpha > 0$. To prove that $C_{\infty}$ is a cone, we take any direction $d$ from $C_{\infty}$. Since d is an element of $C_{\infty}$,
        \begin{equation}
          \exists t_k \rightarrow + \infty , \exists x_k \in C \:\text{ with }\: \lim_{k \to \infty} \frac{x_k}{t_k} = d. \notag
        \end{equation}
        Then we define a sequence $\{t'_k\}_{k=1}^{\infty} \coloneqq \frac{t_k}{\alpha}$, exactly whose limit becomes $+\infty$ as $k \rightarrow \infty$. Accordingly there exists $t'_k \rightarrow + \infty$ and $x_k \in C$ with
        \begin{equation}
          \lim_{k \to \infty} \frac{x_k}{t'_k} = \lim_{k \to \infty} \alpha \cdot \frac{x_k}{t_k} = \alpha d. \notag
        \end{equation}
        It means $d \in C_{\infty}$. By these results, we can get $\forall \alpha \geq 0, d \in C_{\infty}, \alpha d \in C_{\infty}$. Therefore $C_{\infty}$ is a cone.

        \item We show that $C_{\infty}$ is a closed set. In order to prove the closeness, we consider convergency of a sequence in $C_{\infty}$. Firstly, we take a sequence $\{d_k\}_{k=1}^{\infty} \subset C_{\infty}$ with $d_k \rightarrow d$ as $k \rightarrow \infty$ for some $d$. To obtain $d \in C_{\infty}$, we need two sequences such as $\{t_n\}_{n=1}^{\infty}$ with $t_n \rightarrow \infty$ as $n \rightarrow \infty$ and $\{x_n\}_{n=1}^{\infty}$ where $\frac{x_n}{t_n} \rightarrow d$ as $n \rightarrow \infty$.

        Since $d_k \rightarrow d$ and ${t_k^m}^{-1} \cdot x_k^m \rightarrow d_k$ as $m \rightarrow \infty$ for each $k \in \NaturalNumberSet$,
        \begin{equation}
          \begin{split}
            \forall n &\in \NaturalNumberSet, \exists k(n) \in \NaturalNumberSet \SuchThat \forall j \geq k(n), ||d_j -d|| < \frac{1}{n} \\
              &\text{and}\\
            \forall k &\in \NaturalNumberSet (1 \leq k \leq k(n)), \exists m(n,k) \in \NaturalNumberSet \SuchThat \\
            &\forall m \geq m(n,k), ||\frac{x_k^m}{t_k^m} -d_k|| < \frac{1}{n}. \notag
          \end{split}
        \end{equation}
        Now we can rearrange
        \begin{equation}
          \begin{split}
          k(n) &\coloneqq \max\{k(n-1),k(n)\} + n \:\text{and}\: \\
          m(n) &\coloneqq \max_{1 \leq k \leq k(n)}\{m(n,k)\} + n \notag
          \end{split}
        \end{equation}
        as sequences of $n \in \NaturalNumberSet$. Then it holds that $k(1) \leq k(2) \leq \ldots$ and $m(1) \leq m(2) \leq \ldots$. We define
        \begin{equation}
          \begin{split}
            t_n &\coloneqq t_{k(n)}^{m(n)} \:\text{and}\: \\
            x_n &\coloneqq x_{k(n)}^{m(n)}. \notag
          \end{split}
        \end{equation}
        Soon we can get the following results:
        \begin{equation}
          \begin{split}
            t_n &\rightarrow \infty \:\text{as}\: n \rightarrow \infty, \\
            x_n &\in C, \:\text{and}\: \\
            \frac{x_n}{t_n} &= \frac{x_{k(n)}^{m(n)}}{t_{k(n)}^{m(n)}}. \notag
          \end{split}
        \end{equation}
        Hence we obtain for each $n \in \NaturalNumberSet$
        \begin{equation}
          0 \leq \left\lVert \frac{x_n}{t_n} - d\right\rVert \leq \left\lVert \frac{x_{k(n)}^{m(n)}}{t_{k(n)}^{m(n)}} - d_{k(n)}\right\rVert + \left\lVert d_{k(n)} -d \right\rVert< \frac{1}{2n} \rightarrow 0 \notag
        \end{equation}
        as $n \rightarrow \infty$.
        Thus $d \in C_{\infty}$, that is, $C_{\infty}$ is a closed set.
      \end{enumerate}
      Then the proof of (i) is completed.
    \item $(\text{cl}\:C)_{\infty} = C_{\infty}$.

    We need to show two relations: (ii-a) $(\text{cl}\:C)_{\infty} \supset C_{\infty}$ (ii-b) $(\text{cl}\:C)_{\infty} \subset C_{\infty}$.

      \begin{enumerate}[label=ii-\alph*,align=CenterWithParen2]
        \item We show that $C_{\infty}$ is included in $(\text{cl}\:C)_{\infty}$. However it is clear from the definition of asymptotic cones.

        \item We show that $(\text{cl}\:C)_{\infty} \subset C_{\infty}$. Like (i-b), we will show that.

      \end{enumerate}
      Then the proof of (ii) is also completed.
    \item If $C$ is a cone, then $C_{\infty} = \text{cl}\:C$.

    We need to show two relations: (iii-a) $C_{\infty} \subset \text{cl}\:C$ and (iii-b) $C_{\infty} \supset  \text{cl}\:C$.

    \begin{enumerate}[label=iii-\alph*,align=CenterWithParen2]
      \item We take any direction $d \in C_{\infty}$ which satisfies
      \begin{equation}
        \exists t_k \rightarrow + \infty , \exists x_k \in C \:\text{ with }\: \lim_{k \to \infty} \frac{x_k}{t_k} = d. \notag
      \end{equation}
      Let $d_k \coloneqq \frac{x_k}{t_k}$ (with $d_k \rightarrow d$ as $k \rightarrow \infty$). Since $C$ is a cone,
      \begin{equation}
        d_k = \frac{1}{t_k} \cdot x_k \in C. \notag
      \end{equation}
      Due to $d_k \in C$, the limit of $d_k$ is an element of $\text{cl}\:C$, i.e., $d \in \text{cl}\:C$.
      Therefore $C_{\infty} \subset \text{cl}\:C$.

      \item We take any $d \in \text{cl}\:C$ and show $d \in C_{\infty}$, that is,
      \begin{equation}
        \exists t_k \rightarrow + \infty , \exists x_k \in C \:\text{ with }\: \lim_{k \to \infty} \frac{x_k}{t_k} = d. \notag
      \end{equation}
      By $d \in \text{cl}\:C$,
      \begin{equation}
        \exists \{d_k\}_{k=1}^{\infty} \in C \:\text{with}\: d_k \rightarrow d \:\text{as}\: k \rightarrow \infty, \notag
      \end{equation}
      in other words,
      \begin{equation}
        \lim_{k \to \infty} d_k = d. \notag
      \end{equation}
      We define $y_k = k \cdot d_k$ and $s_k = k$ for each $k$. Since $d_k \in C$ and $C$ is a cone, $y_k$ is also a element of $C$.

      There exist $\{s_k\}_{k=1}^{\infty}$ with $s_k \rightarrow \infty$ as $k \rightarrow \infty$ and $\{y_k\}_{k=1}^{\infty} \subset C$ such that
      \begin{equation}
        \lim_{k \to \infty} \frac{y_k}{s_k} = \lim_{k \to \infty} d_k = d. \notag
      \end{equation}
      As $d \in C_{\infty}$, therefore $C_{\infty} \supset  \text{cl}\:C$.

    \end{enumerate}
  \end{enumerate}
\end{proof}

\begin{prop}
  A set $C \subset \mathbb{R}^n$ is bounded if and only if $C_\infty = \{0\}$.
\end{prop}

\begin{proof}
  We show that:

  \begin{enumerate}[label=\roman*,align=CenterWithParen]
    \item a set $C \subset \NDemenstionalRealEuclideanSpace$ is bounded $\Rightarrow $ $C_{\infty} = \{0\}$, and
    \item a set $C \subset \NDemenstionalRealEuclideanSpace$ is unbounded $\Rightarrow $ $C_{\infty} \ne \{0\}$.
  \end{enumerate}

  \begin{enumerate}[label=\roman*,align=CenterWithParen]
    \item By Proposition 2.1.1 (i), $0 \in C_{\infty}$. Also, by the assumption $C$ is bounded,
    \begin{equation}
      \exists r > 0, \forall x_k \in C \:\text{where}\: k \in \NaturalNumberSet,  \left\lVert x_k \right\rVert \leq r. \notag
    \end{equation}
    For any sequence $\{t_k\}_{k=1}^{\infty}$ such that $t_k \rightarrow \infty$ as $k \rightarrow \infty$,
    \begin{equation}
      \lim_{k \to \infty} \frac{x_k}{t_k} = 0. \notag
    \end{equation}
    Thus the limit becomes only 0 for any $\{x_k\}_{k=1}^{\infty} \subset C$ and $\{t_k\}_{k=1}^{\infty}$ with $t_k \rightarrow \infty$ as $k \rightarrow \infty$. Therefore if $C$ is bounded then $C_{\infty} = \{0\}$.

    \item If $C$ is unbounded, then there exists a sequence $\{x_k\} \subset C$ with $x_k \ne 0$, $\forall k \in \NaturalNumberSet$, such that $t_k \coloneqq \left\lVert t_k \right\rVert \rightarrow \infty$ and thus the vectors $d_k = t_k^{-1} x_k \in \{ d \:\colon\: \left\lVert d\right\rVert = 1 \}$. By the Bolzano-Weierstrass, we can extract a subsequence of $\{d_k\}$ such that $\lim_{k \in K} d_k = d$, $K \subset \NaturalNumberSet$, and with $\left\lVert d \right\rVert = 1$. This nonzero vector $d$ is an element of $C_{\infty}$ by Definition 2.1.2, a contradiction.
  \end{enumerate}
\end{proof}

\begin{dfn}
  Let $C \subset \NDemenstionalRealEuclideanSpace$ be nonempty and define
  \begin{equation}
    C_{\infty}^1 = \left\{d \in \NDemenstionalRealEuclideanSpace \:\middle|\: \forall t_k \rightarrow + \infty , \exists x_k \in C \:\text{ with }\: \lim_{k \to \infty} \frac{x_k}{t_k} = d \right\}. \notag
  \end{equation}
  We say that $C$ is asymptotically regular if $C_{\infty} = C_{\infty}^1$.
\end{dfn}

\begin{prop}\label{ConvexSetIsAsymptoticallyRegular}
  Let $C$ be a nonempty convex set in $\NDemenstionalRealEuclideanSpace$. Then $C$ is asymptotically regular.
\end{prop}

\begin{proof}
  The inclusion $C^1_{\infty} \subset C_{\infty}$ clearly holds from the definitions of $C^1_{\infty}$ and $C_{\infty}$, respectively. We now show the reverse inclusion $C^1_{\infty} \supset C_{\infty}$. Let $d \in C_{\infty}$. Then, by the definition of asymptotic cones,
  \begin{equation}
    \exists \{x_k\} \in C, \exists \{s_k\} \to +\infty \SuchThat d = \lim_{k \to \infty} \frac{x_k}{s_k}. \notag
  \end{equation}
  Let $x \in C$ and define $d_k = s_k^{-1}(x_k-x)$. Then we have
  \begin{equation}
    d = \lim_{k \to \infty} d_k \quad \text{and} \quad x + s_k d_k \in C. \notag
  \end{equation}
  Now let $\{t_k\}$ be an arbitrary sequence such that $\lim_{k \to \infty} t_k = +\infty$. For any fixed $m \in \NaturalNumberSet$, there exists $k(m)$ with $\lim_{m \to \infty} k(m) = +\infty$ such that $t_m \leq s_{k(m)}$, and since $C$ is convex, we have $x'_m \coloneqq x +t_m d_{k(m)} = (1-s_{k(m)}^{-1} t_m)x + s_{k(m)}^{-1} t_m x_{k(m)} \in C$. Hence, $d = \lim_{m \to \infty} t_m^{-1} x'_m$, proving that $d \in C^1_{\infty}$.
\end{proof}

We note that it can be asymptotically regular even if $C$ is not convex. For example, let $C = \SetForm{x \in \NDemenstionalRealEuclideanSpace}{\left\lVert x \right\rVert \leq 1} \cup \SetForm{x \in \NDemenstionalRealEuclideanSpace}{\left\lVert x \right\rVert \geq 2}$. Then $C_{\infty} = C_{\infty}^1 = \NDemenstionalRealEuclideanSpace$.

\begin{prop}
  Let $C$ be a nonempty convex set in $\NDemenstionalRealEuclideanSpace$. Then the asymptotic cone $C_{\infty}$ is a closed convex cone. Moreover, define the following sets:
  \begin{equation}
    \begin{split}
      D(x) &\coloneqq \SetForm{d \in \NDemenstionalRealEuclideanSpace}{x+td \in \Closure{C}, \forall t>0}, \forall x \in C \\
      E &\coloneqq \SetForm{d \in \NDemenstionalRealEuclideanSpace}{\exists x \in C \SuchThat x+td \in \Closure{C}, \forall t>0}, \\
      F &\coloneqq \SetForm{d \in \NDemenstionalRealEuclideanSpace}{d + \Closure{C} \subset \Closure{C}}. \notag
    \end{split}
  \end{equation}
  Then $D(x)$ is in fact independent of $x$, which iw thus now denoted by $D$, and $D = E = F = C_{\infty}$.
\end{prop}

\begin{proof}
  We need to prove the following statements:
  \begin{enumerate}[label=\roman*,align=CenterWithParen]
    \item $C_{\infty}$ is a closed convex cone,
    \item $\forall x \in C, C_{\infty} \subset D(x)$,
    \item $\forall x \in C, D(x) \subset E$,
    \item $E \subset C_{\infty}$,
    \item $C_{\infty} \subset F$, and
    \item $F \subset C_{\infty}$.
  \end{enumerate}
  We now show each statement.
  \begin{enumerate}[label=\roman*,align=CenterWithParen]
    \item We have already shown that $C_{\infty}$ is a closed cone in Proposition \ref{basicPropositionOfAsymptoticCone}. We only show that $C_{\infty}$ is convex. Let $d_1, d_2 \in C_{\infty}$ and $\lambda \in (0,1)$. By Proposition \ref{ConvexSetIsAsymptoticallyRegular}, we have
    \begin{equation}
      \forall t_k \to \infty, \exists x_k, y_k \in C \:\text{with}\: \lim_{k \to \infty} \frac{x_k}{t_k} = d_1 \: \text{and}\: \lim_{k \to \infty} \frac{y_k}{t_k} = d_2. \notag
    \end{equation}
    Then, we obtain
    \begin{equation}
      (1-t)\frac{x_k}{t_k} + t \frac{y_k}{t_k} = \frac{(1-t)x_k+ty_k}{t_k}. \notag
    \end{equation}
    The convexity of $C$ implies that $(1-t)x_k+ty_k \in C$ and hence $C_{\infty}$is convex.
    \item Let $d \in C_{\infty}$, $x \in C$, and $t>0$. From the definition of $C_{\infty}$,
    \begin{equation}
      \exists t_k \to \infty, \exists d_k \to d \quad \text{with} \quad x+ t_k d_k \in C. \notag
    \end{equation}
    Take $k$ sufficiently large such that $t \leq t_k$. Since $C$ is convex, we have
    \begin{equation}
      x + t d_k = (1 - \frac{t}{t_k})x + \frac{t}{t_k}(x + t_k d_k) \in C. \notag
    \end{equation}
    Passing to the limit as $k \to \infty$, we thus obtain $x + td \in \Closure{C}$, $\forall t>0$, proving $C_{\infty} \subset D(x)$.
    \item The inclusion $D(x) \subset E$ is obvious.
    \item We now show that $E \subset C_{\infty}$. For that, let $d \in E$ and $x \in C$ be such that $x(t) \coloneqq x + td \in \Closure{C}, \forall t>0$. Then, since $d = \lim_{t \to \infty} t^{-1}x(t)$ and $x(t) \in \Closure{C}$, we have $d \in (\Closure{C})_{\infty} = C_{\infty}$ by Proposition \ref{basicPropositionOfAsymptoticCone}, which also proves that $D(x) = D = C_{\infty} = E$.
    \item Let $d \in C_{\infty}$. Using the representation $C_{\infty} = D$, we thus have $x+d \in \Closure{C}, \forall x \in C$, and hence $d + \Closure{C} \subset \Closure{C}$, proving the inclusion $C_{\infty} \subset F$.
    \item Now, if $d \in F$, then $d + \Closure{C} \subset \Closure{C}$, and it follows that
    \begin{equation}
      2d + \Closure{C} = d + (d + \Closure{C}) \subset d + \Closure{C} \subset \Closure{C}. \notag
    \end{equation}
    and by induction, $\Closure{C} + md \subset \Closure{C} \forall m\in \NaturalNumberSet$. Therefore, $\forall x \in \Closure{C}$, $d_m \coloneqq x+md \in \Closure{C}$ and $d = \lim_{m \to \infty} m^{-1}d_m$, namely $d \in C_{\infty}$, showing that $F \subset C_{\infty}$.
  \end{enumerate}
\end{proof}

\section{Asymptotic Functions}

\begin{dfn}
  For any proper function $\ExtendedRealValuedFunction{f}{\NDemenstionalRealEuclideanSpace}$, there exists a unique function $\ExtendedRealValuedFunction{f_{\infty}}{\NDemenstionalRealEuclideanSpace}$ associated with $f$, called the asymptotic function, such that $\Epigraph{f_\infty} = (\Epigraph{f})_{\infty}$.
\end{dfn}

\begin{prop}
  For any proper function $\ExtendedRealValuedFunction{f}{\NDemenstionalRealEuclideanSpace}$, we have:
  \begin{enumerate}[label=\roman*,align=CenterWithParen]
    \item $f_{\infty}$ is lsc and positively homogeneous.
    \item $f_{\infty}(0) = 0$ or $f_{\infty}(0) = - \infty$.
    \item If $f_{\infty}(0) = 0$, then $f_{\infty}$ is proper.
  \end{enumerate}
\end{prop}

\begin{proof}
  We show each statement.
  \begin{enumerate}[label=\roman*,align=CenterWithParen]
    \item By the definition of the asymptotic function and the fact that asymptotic cones are closed (see Proposition \ref{basicPropositionOfAsymptoticCone}), it follows that $f_{\infty}$ is l.s.c. \\
    First, note that $0 \in \Domain{f_{\infty}}$. Let $x \in \Domain{f_{\infty}}$. Since $(\Epigraph{f})_{\infty}$ is a cone, we have
    \begin{equation}
      (\lambda x, \lambda f_{\infty}(x)) \in \Epigraph{f_{\infty}}, \quad \forall \lambda > 0. \notag
    \end{equation}
    By the definition of epigraph, we obtain $f_{\infty} (\lambda x) \leq \lambda f_{\infty}(x)$. \\
    Likewise, $\forall x \in \Domain{f_{\infty}}, \lambda > 0$, $(\lambda x, f_{\infty}(\lambda x)) \in \Epigraph{f_{\infty}}$ and hence one has $(x, \lambda^{-1}f_{\infty}(\lambda x)) \in \Epigraph{f_\infty}$ by the cone property. Therefore, $\lambda f_{\infty}(x) \leq f_{\infty} (\lambda x)$
    \item Since $f$ is proper, then $\Epigraph{f}$ is nonempty, and hence either $f_{\infty}(0)$ is finite or $f_{\infty}(0) = - \infty$. If $f_{\infty}(0)$ is finite, then by the result of (i), $f_{\infty}(0) = \lambda f_{\infty}(0), \forall \lambda > 0$. Thus, $f_{\infty}(0) = 0$.
  \item Suppose that $f_{\infty}$ is not proper. Then, by the definition, there exists $x$ such that $f_{\infty}(x) = -\infty$. Now let $\{\lambda_k\} \subset \NDemenstionalRealEuclideanSpace_{++}$ be a positive sequence converging to $0$. Then $\lambda_k x \rightarrow 0$, and under our assumption, the lower semicontinuity of $f_{\infty}$, and the result of (i), we have
  \begin{equation}
    0 = f_{\infty}(0) \leq \liminf_{k \to \infty} f_{\infty}(\lambda_k x) = \liminf_{k \to \infty} \lambda_k f_{\infty}(x) = - \infty, \notag
  \end{equation}
  leading to a contradiction. Therefore, $f_{\infty}$ is proper.
  \end{enumerate}
\end{proof}

We now give a fundamental representation of the asymptotic function.

\begin{thm}
  For any proper function $\ExtendedRealValuedFunction{f}{\NDemenstionalRealEuclideanSpace}$, the asymptotic function $f_{\infty}$ is represented as
  \begin{equation}
    f_{\infty}(d) = \liminf_{t \to +\infty, d' \to d} \frac{f(td')}{t},
  \end{equation}
  or equivalently,
  \begin{equation}
    f_{\infty}(d) = \inf \left\{ \liminf_{k \to \infty} \frac{f(t_k d_k)}{t_k} \:|\: t_k \to +\infty, d_k \to d \right\},
  \end{equation}
  where $\{t_k\}$ and $\{d_k\}$ are sequences in $\RealNumberSet$ and  $\NDemenstionalRealEuclideanSpace$, respectively.
\end{thm}

\begin{proof}
  We show that $(\Epigraph{f})_{\infty} = \Epigraph{g}$, where $g$ is defined by
  \begin{equation}
    g(d) = \inf \left\{ \liminf_{k \to \infty} \frac{f(t_k d_k)}{t_k} \:|\: t_k \to +\infty, d_k \to d \right\}. \notag
  \end{equation}
  To begin with, we prove $(\Epigraph{f})_{\infty} \subset \Epigraph{g}$. For any $(d, \mu) \in (\Epigraph{f})_{\infty}$, by the definition of asymptotic cones, one has
  \begin{equation}
    \exists t_k \to \infty, \exists (d_k, \mu_k) \in \Epigraph{f} \:\text{with}\: \frac{(d_k, \mu_k)}{t_k} \to (d, \mu) \:\text{as}\: k \to \infty. \notag
  \end{equation}
  By the definition of epigraph, for $k$ sufficiently large, we have
  \begin{gather}
    f(d_k) \leq \mu_k \notag\\
    \frac{f(d_k)}{t_k} \leq \frac{\mu_k}{t_k} \notag\\
    \frac{f(t_k \cdot {t_k}^{-1}d_k)}{t_k} \leq \frac{\mu_k}{t_k} \notag\\
    \liminf_{k \to \infty} \frac{f(t_k \cdot {t_k}^{-1}d_k)}{t_k} \leq \liminf_{k \to \infty} \frac{\mu_k}{t_k} = \mu \notag \\
    \inf \left\{ {\liminf}_{k \to \infty} \frac{f(t_k \cdot s_k)}{t_k} \:|\: t_k \to +\infty, s_k \to d \right\}\leq \mu. \notag
  \end{gather}
  It actually means that $(d, \mu)$ is an element of $\Epigraph{g}$, and hence $(\Epigraph{f})_{\infty} \subset \Epigraph{g}$. \\
  Next, we prove the reverse inclusion $\Epigraph{g} \subset (\Epigraph{f})_{\infty}$. Let $(d, \mu) \in \Epigraph{g}$. Then, $\forall \epsilon > 0$, there exist $s_k \to \infty$ and $u_k \to d$ such that
  \begin{equation}
    \liminf_{k \to \infty} \frac{f(s_k u_k)}{s_k} \leq \inf \left\{ {\liminf}_{k \to \infty} \frac{f(t_k \cdot s_k)}{t_k} \:|\: t_k \to +\infty, s_k \to d \right\} + \epsilon. \notag
  \end{equation}
  In addition, for all $k \in \NaturalNumberSet$ sufficiently large, we have
  \begin{equation}
    \frac{f(s_k \cdot u_k)}{s_k} \leq \liminf_{k \to \infty} \frac{f(s_k u_k)}{s_k} + \epsilon. \notag
  \end{equation}
  Furthermore, by the definition of epigraph, we obtain $g(d) \leq \mu$. As for these inequalities, it follows that $f(s_k u_k) \leq (\mu + \epsilon) s_k$ and hence $z_k \coloneqq s_k (u_k,\mu + \epsilon) \in \Epigraph{f}$. Since ${s_k}^{-1}z_k \to (d, \mu + \epsilon)$ as $k \to \infty$, it holds that $(d, \mu + \epsilon) \in (\Epigraph{f})_{\infty}$, and since $(\Epigraph{f})_{\infty}$ is closed and $\epsilon$ is arbitrary, we have $(d, \mu) \in (\Epigraph{f})_{\infty}$, proving $\Epigraph{g} \subset (\Epigraph{f})_{\infty}$.
\end{proof}

\begin{prop}
  Let $\ExtendedRealValuedFunction{f}{\NDemenstionalRealEuclideanSpace}$ be a proper, l.s.c, convex function. The asymptotic function $f_{\infty}$ is positively homogeneous, l.s.c., proper convex function, and for any $d \in \NDemenstionalRealEuclideanSpace$, one has
  \begin{equation}
    f_{\infty}(d) = \sup \{f(x+d) -f(x) \:|\: x \in \Domain{f}\}
  \end{equation}
  and
  \begin{equation}
    f_{\infty}(d) = \lim_{t \to +\infty} \frac{f(x+td) -f(x)}{t} = \sup_{t>0} \frac{f(x+td) - f(x)}{t}, \forall x \in \Domain{f}.
  \end{equation}
\end{prop}

\begin{proof}
  TODO
\end{proof}

\begin{thm}
  Let $\ExtendedRealValuedFunction{f}{\NDemenstionalRealEuclideanSpace}$ be a proper convex function, and $\ConjugateFunction{f}$ its conjugate. The following relations hold:
  \begin{enumerate}[label=\alph*,align=CenterWithParen]
    \item $(\ConjugateFunction{f})_{\infty} = \SupportFunction{\Domain{f}}$.
    \item If $f$ is also assumed l.s.c., then
    \begin{equation}
      f_{\infty} = \SupportFunction{\Domain{\ConjugateFunction{f}}},\quad \ConjugateFunction{f_{\infty}} = \IndicatorFunction{\Closure{\Domain{\ConjugateFunction{f}}}}. \notag
    \end{equation}
  \end{enumerate}
\end{thm}

\begin{proof}
  TODO
\end{proof}

% 5. Application of Asymptotic Functions to Semidefinite Programming
% ----------------------------------------------------------------
\chapter{Application of Asymptotic Functions to Semidefinite Programming}

This chapter can be considered as the main part of the paper. Semidefinite programming (SDP) is one of the most significant optimization problems because these problems have appeared in several areas of mathematical sciences and engineering; and have been expected as a wide rage of application. At first, we introduce some optimization problems and the definition of SDP. After that, we will dig in the relation between SDP and asymptotic functions.

\section{Semidefinite Programming}

Mathematical optimization problems are initially classified into two types, continuous optimization problem and discrete optimization problem. In continuous optimization problem, the variables of the problem are continuos. On the other hand, in the case that the variables can only be taken as discrete values like 0 or 1, the problem is called discrete optimization. General mathematical programming have twe types, convex programming and non-convex programming. Convex programming is a problem of minimizing a convex function subject to convex constraints of the domain. SDP we centrally consider is one of the convex programming problems and has several famous mathematical optimization problems like liner programming, convex quadratic programming, and second-order cone programming.

When we suppose that $C \in \NDemenstionalRealSymmetricMatrixSpace$, $A_i \in \NDemenstionalRealSymmetricMatrixSpace$, $b_i \in \RealNumberSet$, $i = 1,\dots,m$, and $X \in \NDemenstionalRealSymmetricMatrixSpace$, which is a variable on the problem, the following constrained problem is called semidefinite programming (SDP).

\begin{equation}
  \begin{aligned}
  \min \quad & \InnerProduct{C}{X}\\
  \textrm{s.t.} \quad & \InnerProduct{A_i}{X} = b_i, \quad i = 1,\dots,m\\
    & X \succcurlyeq 0. \\ \notag
  \end{aligned}
\end{equation}
\section{Spectrally Defined Matrix Functions}

Spectrally defined matrix functions are often seen in Semidefinite programming. They have strong connection with matrix optimizations, called semidefinite programming. At first, we introduce a key concept of matrix optimization, symmetric functions.

\begin{dfn}
  Let $\ExtendedRealValuedFunction{f}{\NDemenstionalRealEuclideanSpace}$.
  $f$ is said to be symmetric if
  \begin{equation}
    \forall x \in \NDemenstionalRealEuclideanSpace \:\text{and}\: P: n \times n \text{ permutation matrix}, f(Px) = f(x). \notag
  \end{equation}
\end{dfn}
For example, each of the following functions is symmetric.
\begin{enumerate}[label=\roman*,align=CenterWithParen]
  \item $f(x) = \max_{1 \leq i \leq n} x_i$ (or $\min_{1 \leq i \leq n} x_i$),
  \item $f(x) = \sum_{i = 1}^{n} x_i$ (or $\prod_{i = 1}^{n} x_i$).
\end{enumerate}

\begin{dfn}
  The function $\ExtendedRealValuedFunction{\Phi}{\NDemenstionalRealSymmetricMatrixSpace}$ is said to be spectrally defined if there exists a symmetric function $\ExtendedRealValuedFunction{f}{\NDemenstionalRealEuclideanSpace}$ such that
  \begin{equation}
    \Phi (X) = \Phi_{f}(X) \coloneqq f(\lambda (X)), \forall X \in \NDemenstionalRealSymmetricMatrixSpace \notag
  \end{equation}
  where $\lambda (X) \coloneqq (\lambda_1 (X), \dotsb , \lambda_n (X))^T$ is the vector of eigenvalues of $X$ in nondecreasing order.
\end{dfn}

When we define the following symmetric function
\begin{equation}
  f(\lambda)=
    \begin{cases}
      - \Sigma_{i = 1}^{n} \log \lambda_i & \text{\rm if}\:\lambda > 0; \\
      +\infty & \text{\rm otherwise}, \notag
    \end{cases}
\end{equation}
the spectrally defined function is deduced;
\begin{equation}
  \Phi_{f}(X)=
    \begin{cases}
      - \log \det (X) & \text{\rm if}\:X \succ 0; \\
      +\infty & \text{\rm otherwise}. \notag
    \end{cases}
\end{equation}

\begin{prop}
  The $\Phi$ is spectrally defined if and only if $\Phi$ is othonormally in variant, that is,
  \begin{equation}
    \Phi (UAU^T) = \Phi (A), \forall A \in \mathcal{U}_n, \notag
  \end{equation}
  where $\mathcal{U}_n \coloneqq$ the set $n \times n$ orthogonal matrices.
\end{prop}

\begin{proof}
  TODO
\end{proof}

To prove the following key theorem, we introduce two properties of symmetric matrices.

\begin{lem}\label{lemma1ForLewis96}
  $\forall Y \in \NDemenstionalRealSymmetricMatrixSpace$, there exists a orthogonal matrix $U_0$ diagonalizing $Y$ and satisfying
  \begin{equation}
    \forall X \in \NDemenstionalRealSymmetricMatrixSpace, \InnerProduct{X}{Y} = \InnerProduct{U_0^TXU_0}{\Diagnosis{\lambda (Y)}}. \notag
  \end{equation}
\end{lem}

\begin{proof}
  Since Y is an element in $\NDemenstionalRealSymmetricMatrixSpace$, there exists a orthogonal matrix $U_0$ diagonalizing $Y$. Then, for all $X \in \NDemenstionalRealSymmetricMatrixSpace$, we have
  \begin{equation}
    \begin{split}
      \InnerProduct{X}{Y} &= \Trace{XY} = \Trace{XYU_0U_0^T} = \Trace{U_0^TXYU_0} \\
      &= \Trace{U_0^TXU_0U_0^TYU_0}\\
      &= \InnerProduct{U_0^TXU_0}{U_0^TYU_0}\\
      &= \InnerProduct{U_0^TXU_0}{\Diagnosis{\lambda (Y)}}. \notag
    \end{split}
  \end{equation}
\end{proof}

\begin{lem}\label{lemma2ForLewis96}
  Let $B: \NDemenstionalRealSymmetricMatrixSpace \to \NDemenstionalRealSymmetricMatrixSpace$. Suppose to $B(X) = U^TXU$ with $U \in \mathcal{U}_n$ where $\mathcal{U}_n \coloneqq$ the set $n \times n$ orthogonal matrices. Then $U^TXU \in \NDemenstionalRealSymmetricMatrixSpace, \forall X \in \NDemenstionalRealSymmetricMatrixSpace$ and $B$ is surjective.
\end{lem}

\begin{proof}
  $\forall X \in \NDemenstionalRealSymmetricMatrixSpace, U \in \mathcal{U}_n$, since $X$ is symmetric,
  \begin{equation}
    (U^TXU)^T = U^TX^TU = U^TXU. \notag
  \end{equation}
  Thus, $U^TXU$ is an element of $\NDemenstionalRealSymmetricMatrixSpace$. \\
  In addition, we show that  $\forall Y \in \NDemenstionalRealSymmetricMatrixSpace, \exists X \in \NDemenstionalRealSymmetricMatrixSpace \SuchThat B(X) = Y$. For any $Y \in \NDemenstionalRealSymmetricMatrixSpace$, we put $X = UYU^T$. Then, we obtain $B(X) = U^TXU = Y$. Therefore, $B$ is surjective.
\end{proof}

\begin{thm}{(A.S.Lewis (1996), \cite{Lewis96})}
  Suppose that the function $\ExtendedRealValuedFunction{f}{\NDemenstionalRealEuclideanSpace}$ is symmetric, then
  \begin{equation}
    {\Phi_{f}}^* = \Phi_{f^*}. \notag
  \end{equation}
\end{thm}

\begin{proof}
  By the results of Lemma \ref{lemma1ForLewis96} and Lemma \ref{lemma2ForLewis96}, we have $\ConjugateFunction{\Phi_f}(Y) = \ConjugateFunction{\Phi}(\Diagnosis{\lambda (Y)})$. Using the definition of the conjugate functions, one thus has
  \begin{equation}
    \begin{split}
      \ConjugateFunction{\Phi_f}(Y) &= \ConjugateFunction{\Phi}(\Diagnosis{\lambda (Y)})\\
      &= \sup_{X \in \NDemenstionalRealSymmetricMatrixSpace} \{\InnerProduct{X}{\Diagnosis{\lambda (Y)}} - \Phi_f(X)\}\\
      &\geq \sup_{x \in \NDemenstionalRealEuclideanSpace} \{\InnerProduct{\Diagnosis{x}}{\Diagnosis{\lambda (Y)}} - \Phi_f(\Diagnosis{x})\}\\
      &= \sup_{x \in \NDemenstionalRealEuclideanSpace} \{\InnerProduct{x}{\lambda (Y)} - f(x)\}\\
      &= \ConjugateFunction{f}(\lambda (Y)) = \Phi_{\ConjugateFunction{f}}(Y). \notag
    \end{split}
  \end{equation}
  The proof of the reverse inequality $\ConjugateFunction{\Phi_f}(Y) \leq \Phi_{\ConjugateFunction{f}}(Y)$ is obtained by combining the Fenchel-Young inequality $\ConjugateFunction{f}(\lambda(Y)) + f(\lambda(X)) \geq \InnerProduct{\lambda(X)}{\lambda(Y)}$ with the trace inequality. Moving to the details of the proof, it holds that $\forall X \in \NDemenstionalRealSymmetricMatrixSpace$,
  \begin{equation}
    \begin{split}
      \Phi_{\ConjugateFunction{f}}(Y) &\geq \InnerProduct{\lambda(X)}{\lambda(Y)} - f(\lambda(X))\\
      &\geq \InnerProduct{X}{Y} - f(\lambda(X))\\
      &= \InnerProduct{X}{Y} - \Phi_f(X). \notag
    \end{split}
  \end{equation}
  As a result, we obtain $\ConjugateFunction{\Phi_f}(Y) \geq \Phi_{\ConjugateFunction{f}}(Y) = \ConjugateFunction{\Phi_f}(Y)$.
\end{proof}

\begin{prop}
  Let $\ExtendedRealValuedFunction{f}{\NDemenstionalRealEuclideanSpace}$ be a symmetric proper convex and lsc function, $\ExtendedRealValuedFunction{\Phi_f}{\NDemenstionalRealSymmetricMatrixSpace}$ the induced spectral function. Then, $\ConjugateFunction{f}$ is symmetric, and the following relations hold:
  \begin{enumerate}[label=\roman*,align=CenterWithParen]
    \item $\Phi_{f}$ is proper, convex, and lsc.
    \item $\ConjugateFunction{\Phi_f}(Y) + \Phi_f(X) \geq \InnerProduct{\lambda(X)}{\lambda(Y)}, \forall X,Y \in \NDemenstionalRealSymmetricMatrixSpace$.
    \item $Y \in \partial \Phi_f(X) \Leftrightarrow \InnerProduct{\lambda(X)}{\lambda(Y)} = \InnerProduct{X}{Y} and \lambda(Y) \in \partial f(\lambda(x))$.
  \end{enumerate}
\end{prop}

\begin{proof}
  We now prove each statement.
  \begin{enumerate}[label=\roman*,align=CenterWithParen]
    \item It is obvious by the definition of the induced spectral function $\Phi_f$ and the assumption of $f$.
    \item By the definition of the conjugate function, it holds that $\forall Y \in \NDemenstionalRealSymmetricMatrixSpace$,
    \begin{equation}
      \begin{split}
        \ConjugateFunction{\Phi}(Y) &= \sup_{X \in \NDemenstionalRealSymmetricMatrixSpace} \{\InnerProduct{X}{Y} - \Phi(X)\} \\
        &\geq \InnerProduct{X}{Y} - \Phi(X), \quad \forall X \in \NDemenstionalRealSymmetricMatrixSpace. \notag
      \end{split}
    \end{equation}
    Thus, we obtain $\ConjugateFunction{\Phi}(Y) + \Phi(X) \geq \InnerProduct{X}{Y}$.
    \item We show ($\Rightarrow$) and ($\Leftarrow$).

    ($\Rightarrow$) By Fenchel-Young inequality, the definition of the spectral function, the assumption, the result of (ii) and the trace inequality, we have
    \begin{equation}
      \begin{split}
        \InnerProduct{\lambda(X)}{\lambda(Y)} &\leq \ConjugateFunction{f}(\lambda(Y)) + f(\lambda(X)) \\
        &= \ConjugateFunction{\Phi}(Y) + \Phi(X) \\
        &= \InnerProduct{X}{Y} \\
        &\leq \InnerProduct{\lambda(X)}{\lambda(Y)}. \notag
      \end{split}
    \end{equation}
    Additionally, for $X$ and $Y$ satisfying $Y \in \partial \Phi_f(X)$, one has
    \begin{equation}
      \begin{split}
        \Phi(Z) &\geq \Phi(X) + \InnerProduct{Z-X}{Y}, \quad \forall Z \in \NDemenstionalRealSymmetricMatrixSpace \\
        f(\lambda(Z)) &\geq f(\lambda(X)) + \InnerProduct{\lambda(Z)}{\lambda(Y)} - \InnerProduct{\lambda(X)}{\lambda(Y)}, \quad \forall Z \in \NDemenstionalRealSymmetricMatrixSpace\\
        f(z) &\geq f(\lambda(X)) + \InnerProduct{z - \lambda(X)}{\lambda(Y)}, \quad \forall z \in \NDemenstionalRealEuclideanSpace. \notag
      \end{split}
    \end{equation}
    Thus, $\lambda(Y) \in \partial f(\lambda(X))$.

    ($\Leftarrow$) It is clear by the proof of ($\Rightarrow$).
  \end{enumerate}
\end{proof}

As a result, the optimization problems $\min \{\Phi_{f}(X) \:|\: X \in \NDemenstionalRealSymmetricMatrixSpace\}$ and $\min \{f(x) \:|\: x \in \NDemenstionalRealEuclideanSpace\}$ are equivalent. In fact,

\begin{align}
  \inf_{X \in \NDemenstionalRealSymmetricMatrixSpace} \Phi_{f}(X) &= - \sup_{X \in \NDemenstionalRealSymmetricMatrixSpace} \{- \Phi_{f}(X)\} = - \sup_{X \in \NDemenstionalRealSymmetricMatrixSpace} \{\InnerProduct{X}{0} - \Phi_{f}(X)\} \notag \\
  &= - \Phi^*_{f}(0) = - \Phi_{\ConjugateFunction{f}}(0) = - \ConjugateFunction{f}(0) = \inf_{x \in \NDemenstionalRealEuclideanSpace} f(x). \notag
\end{align}

\begin{dfn}
  The asymptotic functions of the proper convex lsc function $\ExtendedRealValuedFunction{\Phi}{\NDemenstionalRealSymmetricMatrixSpace}$ is defined by, for all $D \in \NDemenstionalRealSymmetricMatrixSpace$
  \begin{equation}
    \begin{split}
      \Phi_{\infty} (D) &= \sup_{t > 0} \frac{\Phi(A+tD) -\Phi(A)}{t}, \forall A \in \Domain{\Phi} \quad \text{and} \\
      &= \sup \{\InnerProduct{B}{D} \:|\: B \in \Domain{\ConjugateFunction{\Phi}}\}. \notag
    \end{split}
  \end{equation}
\end{dfn}

\begin{thm}{(A.Seeger (1997), \cite{Seeger97})}
  Let $\ExtendedRealValuedFunction{f}{\NDemenstionalRealEuclideanSpace}$ be a symmetric, lsc, proper, convex function with induced spectral function $\Phi_{f}$. Then
  \begin{equation}
    {(\Phi_{f})}_{\infty} = \Phi_{f_{\infty}}. \notag
  \end{equation}
\end{thm}

\begin{proof}
  TODO
\end{proof}
% 6. Conclusion
% ----------------------------------------------------------------
\chapter{Conclusion}

\begin{thebibliography}{9}
  \bibitem{Auslender03}
  A. Auslender and M. Teboulle, Asymptotic cones and functions in optimization and variational inequalities, Springer monographs in Mathematics, Springer-Verlag, New York, 2003.

  \bibitem{Auslender03}
  J.M. Borwein and A.S. Lewis, Convex Analysis and Nonlinear Optimization: Theory and Examples, Springer-Verlag, New York, 2000.

  \bibitem{Tammer03}
  A. G\"{o}pfert, H. Riahi, C. Tammer, and C. Z\u{a}linescu, Variational methods in partially ordered spaces, vol. 17 of CMS Books in Mathematics, Springer-Verlag, New York, 2003.

  \bibitem{Lewis96}
  A.S. Lewis. Convex Analysis on the Hermitian matrices. SIAM J. Optimization,6, 1996, 164-177.

  \bibitem{Seeger97}
  A. Seeger. Convex analysis of spectrally defined matrix functions. SIAM J. Optimization , 7, 1997, 679-696.

  \end{thebibliography}
\end{document}

\end{document}
