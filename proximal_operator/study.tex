\documentclass[a4paper,11pt]{jsarticle}

% 英文化
\usepackage[english]{babel}
% 数式
\usepackage{amsmath}
\usepackage{amsfonts}
\usepackage{amsthm}
\usepackage{amssymb}
\usepackage{bm}
\usepackage{mathtools}
% 画像
\usepackage[dvipdfmx]{graphicx}
% 箇条書き
\usepackage{enumitem}

% 枠付き文章
\usepackage{ascmac}

\newcommand{\MiniBox}[1]{\fbox{
  \begin{minipage}{.8\textwidth}
    #1
  \end{minipage}
}}

\SetLabelAlign{Center}{\hfil#1\hfil}
\SetLabelAlign{CenterWithParen}{\hfil(\makebox[1.0em]{#1})\hfil}

\newtheorem{thm}{Theorem}[section]
\newtheorem{prop}[thm]{Proposition}
\newtheorem{lem}[thm]{Lemma}
\newtheorem{cor}[thm]{Corollary}
\newtheorem{conj}[thm]{Conjecture}
\theoremstyle{definition}
\newtheorem{dfn}[thm]{Definition}
\newtheorem{rem}[thm]{Remark}
\newtheorem{fact}[thm]{Fact}


% Mathematical Sets
\newcommand{\NaturalNumberSet}{\mathbb{N}}
\newcommand{\RealNumberSet}{\mathbb{R}}
\newcommand{\NDemenstionalRealEuclideanSpace}{\mathbb{R}^n}
\newcommand{\MDemenstionalRealEuclideanSpace}{\mathbb{R}^m}
\newcommand{\NDemenstionalRealSymmetricMatrixSpace}{\mathbb{S}^n}
\newcommand{\NDemenstionalRealOthonormalMatrixSpace}{\mathcal{U}_n}


% Symbols like prefix
\newcommand{\Closure}[1]{\text{\rm cl\:${#1}$}} % cl
\newcommand{\Interior}[1]{\text{\rm int\:${#1}$}} % int
\newcommand{\Domain}[1]{\text{\rm dom\:${#1}$}} % dom
\newcommand{\Epigraph}[1]{\text{\rm epi\:${#1}$}} % epi
\newcommand{\KEpigraph}[1]{\text{\rm epi$_K$\:${#1}$}} % epi_K
\newcommand{\LevelSets}[2]{\text{\rm lev\:$({#1}, {#2})$}} % lev
\newcommand{\Trace}[1]{\text{\rm tr$({#1})$}} % tr
\newcommand{\Diagnosis}[1]{\text{\rm diag\:${#1}$}} % diag
\newcommand{\InnerProduct}[2]{\left\langle {#1},{#2}\right\rangle} % <x,y>
\newcommand{\Norm}[1]{\left\lVert {#1} \right\rVert} % ||x||

% Extended real valued function e.g. f: X -> Rv{+∞}
% #1: function symbol
% #2: domain of function
\newcommand{\ExtendedRealValuedFunction}[2]{{#1}: {#2} \to \RealNumberSet \cup \{+\infty\}}
\newcommand{\VectorValuedFunction}[3]{{#1}\:\colon{#2} \to {#3}}


% Conjugate function e.g. f*
% #1: function symbol
\newcommand{\ConjugateFunction}[1]{{#1}^*}

% Support function e.g. f*
% #1: set symbol
\newcommand{\SupportFunction}[1]{\sigma_{#1}}

% Indicator function e.g. f*
% #1: set symbol
\newcommand{\IndicatorFunction}[1]{\delta_{#1}}

% (Useful) Texts
\newcommand{\SuchThat}{\:\text{\rm s.t.}\:}

% Set form e.g. {x | ...}
% #1: element
% #2: conditions
\newcommand{\SetForm}[2]{
  \{{#1}\:|\:{#2}\}
}

\begin{document}

\title{%
  Review: A Semi-Bregman Proximal Alternating Method for a Class of Nonconvex Problems: Local and Global Convergence Analysis}
\author{Ryota Iwamoto}
\date{\today}
\maketitle

\section{Introduction}

We consider the following non-convex and non-smooth block optimization model:

\begin{equation}
  \begin{aligned}
    \min_{x \in \RealNumberSet^n} & \quad \Psi(x,y) \coloneqq  F(x) + \Phi(y) + Q(x,y)                         \\
    \text{s.t.}                   & \quad x \in \NDemenstionalRealEuclideanSpace, \quad y \in \RealNumberSet^m
  \end{aligned}
\end{equation}

In the above model, $F$ and $\Phi$ are non-smooth functions, and $Q$ is a smooth function.
This model has been studied in various fields in recent years, mostly non-convex optimization.
In this paper, we focus our attention on the following class of smooth coupling functions:

\begin{equation}
  Q(x,y) = \frac{\rho}{2} \Norm{q(x) - y}
\end{equation}

where $\rho > 0$ and $q \colon \NDemenstionalRealEuclideanSpace \to \RealNumberSet^m$ is
a continuously differentiable mapping.


\begin{thebibliography}{99}
  \bibitem{Teboulle2024}
  M. Teboule, E. Cohen, D. R. Luke, T. Pinta, and S. Sabach.
  A Semi-Bregman Proximal Alternating Method for a Class of Nonconvex Problems: Local and Global Convergence Analysis.
  Jonunal of Global Optimization, Springer, 89 (2024), 33--55.

  \bibitem{Bolte2014}
  J. Bolte, S. Sabach, and M. Teboulle.
  Proximal alternating linearized minimization for nonconvex and nonsmooth problems.
  Math. Program., 146(1-2):459--494, 2014.

  \bibitem{Bolte2018}
  J. Bolte, S. Sabach, M. Teboulle, and Y. Vasibourd.
  First Order Methods Beyond Convexity and Lipschitz Gradient Continuity with Applications to Quadratic Inverse Problems.
  SIAM J. Optim., 28(3);2131--2151, 2018

  \bibitem{BenTalTeboulle89}

  \bibitem{BorweinLewis00}
  J.M. Borwein and A.S. Lewis. Convex Analysis and Nonlinear Optimization: Theory and Examples, Springer-Verlag, New York, 2000.

  \bibitem{Lewis96}
  A.S. Lewis. Convex Analysis on the Hermitian matrices. SIAM J. Optimization, 6 (1996), 164--177.

  \bibitem{Rockafellar70}
  R.T. Rockafellar. Convex Analysis. Princeton University Press, Princeton, New Jersey, 1970.

  \bibitem{Rockafellar98}
  R.T. Rockafellar and R.J.B Wets. Variational Analysis. Springer-Verlag, New York, 1998.

  \bibitem{Seeger97}
  A. Seeger. Convex analysis of spectrally defined matrix functions. SIAM J. Optimization, 7 (1997), 679--696.

  \bibitem{Flores2024}
  F. Flores-Baz\'{a}n, R. L\'{o}pez and C. Vera. Vector Asymptotic Functions and Their Application to Multiobjective Optimization Problems. SIAM J. Optimization, 34 (2024), 1826--1851.

  \bibitem{Tanaka1995}
  T. Tanaka. Cone-Quasiconvexity of Vector-Valued Functions. Science Reports of the Hirosaki University, 42 (1995), 157--163.

  \bibitem{Tanaka1996}
  T. Tanaka. Approximately Efficient Solutions in Vector Optimization. J. Multi-Criteria Decision Analysis, 5 (1996), 271--278.

  \bibitem{Bot2009}
  R.I. Bo\c{t}, S.-M. Grad and G. Wanka. Duality in Vector Optimization, Springer-Verlag, Berlin-Heidelberg (2009)
\end{thebibliography}

\end{document}