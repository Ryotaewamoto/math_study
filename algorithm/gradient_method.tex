\documentclass[a4paper,11pt]{jsarticle}

% 英文化
\usepackage[english]{babel}
% 数式
\usepackage{amsmath}
\usepackage{amsfonts}
\usepackage{amsthm}
\usepackage{amssymb}
\usepackage{bm}
\usepackage{mathtools}
% 画像
\usepackage[dvipdfmx]{graphicx}
% 箇条書き
\usepackage{enumitem}

% 枠付き文章
\usepackage{ascmac}

\newcommand{\MiniBox}[1]{\fbox{
  \begin{minipage}{.8\textwidth}
    #1
  \end{minipage}
}}

\SetLabelAlign{Center}{\hfil#1\hfil}
\SetLabelAlign{CenterWithParen}{\hfil(\makebox[1.0em]{#1})\hfil}

\newtheorem{thm}{Theorem}[section]
\newtheorem{prop}[thm]{Proposition}
\newtheorem{lem}[thm]{Lemma}
\newtheorem{cor}[thm]{Corollary}
\newtheorem{conj}[thm]{Conjecture}
\theoremstyle{definition}
\newtheorem{dfn}[thm]{Definition}
\newtheorem{rem}[thm]{Remark}
\newtheorem{fact}[thm]{Fact}


% Mathematical Sets
\newcommand{\NaturalNumberSet}{\mathbb{N}}
\newcommand{\RealNumberSet}{\mathbb{R}}
\newcommand{\NDemenstionalRealEuclideanSpace}{\mathbb{R}^n}
\newcommand{\MDemenstionalRealEuclideanSpace}{\mathbb{R}^m}
\newcommand{\NDemenstionalRealSymmetricMatrixSpace}{\mathbb{S}^n}
\newcommand{\NDemenstionalRealOthonormalMatrixSpace}{\mathcal{U}_n}


% Symbols like prefix
\newcommand{\Closure}[1]{\text{\rm cl\:${#1}$}} % cl
\newcommand{\Interior}[1]{\text{\rm int\:${#1}$}} % int
\newcommand{\Domain}[1]{\text{\rm dom\:${#1}$}} % dom
\newcommand{\Epigraph}[1]{\text{\rm epi\:${#1}$}} % epi
\newcommand{\KEpigraph}[1]{\text{\rm epi$_K$\:${#1}$}} % epi_K
\newcommand{\LevelSets}[2]{\text{\rm lev\:$({#1}, {#2})$}} % lev
\newcommand{\Trace}[1]{\text{\rm tr$({#1})$}} % tr
\newcommand{\Diagnosis}[1]{\text{\rm diag\:${#1}$}} % diag
\newcommand{\InnerProduct}[2]{\left\langle {#1},{#2}\right\rangle} % <x,y>
\newcommand{\Norm}[1]{\left\lVert {#1} \right\rVert} % ||x||

% Extended real valued function e.g. f: X -> Rv{+∞}
% #1: function symbol
% #2: domain of function
\newcommand{\ExtendedRealValuedFunction}[2]{{#1}: {#2} \to \RealNumberSet \cup \{+\infty\}}
\newcommand{\VectorValuedFunction}[3]{{#1}\:\colon{#2} \to {#3}}


% Conjugate function e.g. f*
% #1: function symbol
\newcommand{\ConjugateFunction}[1]{{#1}^*}

% Support function e.g. f*
% #1: set symbol
\newcommand{\SupportFunction}[1]{\sigma_{#1}}

% Indicator function e.g. f*
% #1: set symbol
\newcommand{\IndicatorFunction}[1]{\delta_{#1}}

% (Useful) Texts
\newcommand{\SuchThat}{\:\text{\rm s.t.}\:}

% Set form e.g. {x | ...}
% #1: element
% #2: conditions
\newcommand{\SetForm}[2]{
  \{{#1}\:|\:{#2}\}
}

\begin{document}

\title{%
  Review: A Semi-Bregman Proximal Alternating Method for a Class of Nonconvex Problems: Local and Global Convergence Analysis}
\author{Ryota Iwamoto}
\date{\today}
\maketitle

In this section we discuss methods for solving the unconstrained optimization
problem below:

\begin{equation}
  \label{eq:unconstrained_optimization}
  \min_{x \in \RealNumberSet^n} f(x),
\end{equation}

where $f: \RealNumberSet^n \to \RealNumberSet$ is a continuously differentiable (which implies $\Domain{f}$ is open).

\section{Line Search}





\begin{thebibliography}{99}
  \bibitem{Teboulle2024}
  M. Teboule, E. Cohen, D. R. Luke, T. Pinta, and S. Sabach.
  A Semi-Bregman Proximal Alternating Method for a Class of Nonconvex Problems: Local and Global Convergence Analysis.
  Jonunal of Global Optimization, Springer, 89 (2024), 33--55.

  \bibitem{Bolte2014}
  J. Bolte, S. Sabach, and M. Teboulle.
  Proximal alternating linearized minimization for nonconvex and nonsmooth problems.
  Math. Program., 146(1-2):459--494, 2014.

  \bibitem{Bolte2018}
  J. Bolte, S. Sabach, M. Teboulle, and Y. Vasibourd.
  First Order Methods Beyond Convexity and Lipschitz Gradient Continuity with Applications to Quadratic Inverse Problems.
  SIAM J. Optim., 28(3);2131--2151, 2018

\end{thebibliography}

\end{document}