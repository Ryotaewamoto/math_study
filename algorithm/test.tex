\documentclass{jsarticle}
\usepackage{algorithmicx}
\usepackage[ruled]{algorithm}
\usepackage{algpseudocode}
\usepackage{algpascal}
\usepackage{algc}

\newcommand{\alg}{\texttt{algorithmicx}}
\newcommand{\old}{\texttt{algorithmic}}
\newcommand{\euk}{Euclid}
\newcommand\ASTART{\bigskip\noindent\begin{minipage}[b]{0.5\linewidth}}
\newcommand\ACONTINUE{\end{minipage}\begin{minipage}[b]{0.5\linewidth}}
\newcommand\AENDSKIP{\end{minipage}\bigskip}
\newcommand\AEND{\end{minipage}}



\title{The \alg\ package\footnote{This is the documentation for the version 1.2
of the package. This package is released under LPPL.}}
\author{Sz\'asz J\'anos\\szaszjanos@users.sourceforge.net}



\begin{document}
\maketitle
\begin{abstract}
The \alg\ package provides many possibilities to customize the layout of algorithms. 
You can use one of the predefined layouts (\textbf{pseudocode}, \textbf{pascal} 
and \textbf{c} and others), with or without modifications, or you can define a 
completely new layout for your specific needs.
\end{abstract}
\tableofcontents




\section{Introduction}
All this has begun in my last year at the university. The only thing that I knew of 
\LaTeX\ was that it exists, and that it is ``good''. I started using it, but I needed to typeset some 
algorithms. So I begun searching for a good algorithmic style, and I have found the \old\ package. 
It was a great joy for me, and I started to use it\dots\ 
Well\dots\ Everything went nice, until I needed some block that wasn't defined in there. What to do? 
I was no \LaTeX\ guru, in fact I only knew the few basic macros. But there was no other way, so I opened 
the style file, and I copied one existing block, renamed a few things, and voil\`a! This (and some other 
small changes) where enough for me\dots

One year later --- for one good soul --- I had to make some really big changes on the style. And there on 
a sunny day came the idea. What if I would write some macros to let others create blocks automatically? 
And so I did! Since then the style was completely rewritten\dots\ several times\dots

I had fun writing it, may you have fun using it! I am still no \LaTeX\ guru, so if you are, and you find 
something really ugly in the style, please mail me! All ideas for improvements are welcome!

Thanks go to Benedek Zsuzsa, Ionescu Clara, Sz\H ocs Zolt\'an, Cseke Botond, Kanoc 
%Szotyori Zolt\'an Csaba
and many-many others. Without them I would have never started or continued \textbf{algorithmicx}.




\section{General informations}

\subsection{The package}
The package \textbf{algorithmicx} itself doesn't define any algorithmic commands, but gives 
a set of macros to define such a command set. You may use only \textbf{algorithmicx}, and define 
the commands yourself, or you may use one of the predefined command sets.

These predefined command sets (layouts) are:
\begin{description}
\item[algpseudocode] has the same look\footnote{almost :-)} as the one defined in the
\old\ package. The main difference is that while the \old\ package doesn't 
allow you to modify predefined structures, or to create new ones, the \alg\ 
package gives you full control over the definitions (ok, there are some 
limitations --- you can not send mail with a, say, \verb:\For: command).
\item[algcompatible] is fully compatible with the \old\ package, it should be
used only in old documents.
\item[algpascal] aims to create a formatted pascal program, it performs 
automatic indentation (!), so you can transform a pascal program into an 
\textbf{algpascal} algorithm description with some basic substitution rules.
\item[algc] -- yeah, just like the \textbf{algpascal}\dots\ but for c\dots\ 
This layout is incomplete.
\end{description}

To create floating algorithms you will need \verb:algorithm.sty:. This file may or may not be
included in the \alg\ package. You can find it on CTAN, in the \old\ package.



\subsection{The algorithmic block}
Each algorithm begins with the \verb:\begin{algorithmic}[lines]: command, the 
optional \verb:lines: controls the line numbering: $0$ means no line numbering, 
$1$ means number every line, and $n$ means number lines $n$, $2n$, $3n$\dots\ until the
\verb:\end{algorithmic}: command, witch ends the algorithm.



\subsection{Simple lines}

A simple line of text is beginned with \verb:\State:. This macro marks the begin of every 
line. You don't need to use \verb:\State: before a command defined in the package, since 
these commands use automatically a new line.

To obtain a line that is not numbered, and not counted when counting the lines for line numbering
(in case you choose to number lines), use the \verb:Statex: macro. This macro jumps into a new line, 
the line gets no number, and any label will point to the previous numbered line.

We will call \textit{statament\/}s the lines starting with \verb:\State:. The \verb:\Statex: 
lines are not stataments.



\subsection{Placing comments in sources}\label{Putting comments in sources}

Comments may be placed everywhere in the source using the \verb:\Comment: macro 
(there are no limitations like those in the \old\ package), feel the freedom!
If you would like to change the form in witch comments are displayed, just 
change the \verb:\algorithmiccomment: macro:

\begin{verbatim}
\algrenewcommand{\algorithmiccomment}[1]{\hskip3em$\rightarrow$ #1}
\end{verbatim}

will result:
\medskip
\begin{algorithmic}[1]
\algrenewcommand{\algorithmiccomment}[1]{\hskip3em$\rightarrow$ #1}
\State $x\gets x+1$\Comment{Here is the new comment}
\end{algorithmic}



\subsection{Labels and references}
Use the \verb:\label: macro, as usual to label a line. When you use \verb:\ref: to reference 
the line, the \verb:\ref: will be subtitued with the corresponding line number. When using the 
\textbf{algorithmicx} package togedher with the \textbf{algorithm} package, then you can label 
both the algorithm and the line, and use the \verb:\algref: macro to reference a given line 
from a given algorithm:

\begin{verbatim}
\algref{<algorithm>}{<line>}
\end{verbatim}

\noindent\begin{minipage}[t]{0.5\linewidth}
\begin{verbatim}
The \textbf{while} in algorithm
\ref{euclid} ends in line
\ref{euclidendwhile}, so
\algref{euclid}{euclidendwhile}
is the line we seek.
\end{verbatim}
\end{minipage}\begin{minipage}[t]{0.5\linewidth}
The \textbf{while} in algorithm \ref{euclid} ends in line \ref{euclidendwhile},
so \algref{euclid}{euclidendwhile} is the line we seek.
\end{minipage}

\subsection{Breaking up long algorithms}

Sometimes you have a long algorithm that needs to be broken into parts, each on a
separate float. For this you can use the following:

\begin{description}
\item[]\verb:\algstore{<savename>}: saves the line number, indentation, open blocks of
the current algorithm and closes all blocks. If used, then this must be the last command
before closing the algorithmic block. Each saved algorithm must be continued later in the
document.
\item[]\verb:\algstore*{<savename>}: Like the above, but the algorithm must not be continued.
\item[]\verb:\algrestore{<savename>}: restores the state of the algorithm saved under
\verb:<savename>: in this algorithmic block. If used, then this must be the first command
in an algorithmic block. A save is deleted while restoring.
\item[]\verb:\algrestore*{<savename>}: Like the above, but the save will not be deleted, so it
can be restored again.
\end{description}

See example in the \textbf{Examples} section.

\subsection{Multiple layouts in the same document}

You can load multiple algorithmicx layouts in the same document. You can switch between the layouts
using the \verb:\alglanguage{<layoutname>}: command. After this command all new algorithmic
environments will use the given layout until the layout is changed again.




\section{The predefined layouts}

\subsection{The \textbf{algpseudocode} layout}\label{algpseudocode}
\alglanguage{pseudocode}
If you are familiar with the \old\ package, then you'll find it easy to 
switch. You can use the old algorithms with the \textbf{algcompatible} layout, but please
use the \textbf{algpseudocode} layout for new algorithms.

To use \textbf{algpseudocode}, simply load \verb:algpseudocode.sty::

\begin{verbatim}
\usepackage{algpseudocode}
\end{verbatim}

You don't need to manually load the \textbf{algorithmicx} package, as this is done by
\textbf{algpseudocode}.

The first algorithm one should write is the first algorithm ever (ok, 
an improved version), \textit{\euk's algorithm}:

\begin{algorithm}[H]
\caption{\euk's algorithm}\label{euclid}
\begin{algorithmic}[1]
\Procedure{\euk}{$a,b$}\Comment{The g.c.d. of a and b}
   \State $r\gets a\bmod b$
   \While{$r\not=0$}\Comment{We have the answer if r is 0}
      \State $a\gets b$
      \State $b\gets r$
      \State $r\gets a\bmod b$
   \EndWhile\label{euclidendwhile}
   \State \Return $b$\Comment{The gcd is b}
\EndProcedure
\end{algorithmic}
\end{algorithm}

\end{document}