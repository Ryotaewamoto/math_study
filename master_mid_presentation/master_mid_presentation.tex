% \documentclass[dvipdfmx, 11pt]{beamer}
\documentclass[aspectratio=169, dvipdfmx, 11pt]{beamer} % aspectratio=43, 149, 169
\usepackage{here, amsmath, latexsym, amssymb, bm, ascmac, mathtools, multicol, tcolorbox, subfig}

%デザインの選択(省略可)
\usetheme{Luebeck}
%カラーテーマの選択(省略可)
\usecolortheme{orchid}
%フォントテーマの選択(省略可)
\usefonttheme{professionalfonts}
%フレーム内のテーマの選択(省略可)
\useinnertheme{circles}
%フレーム外側のテーマの選択(省略可)
\useoutertheme{infolines}
%しおりの文字化け解消
\usepackage{atbegshi}
\ifnum 42146=\euc"A4A2
\AtBeginShipoutFirst{\special{pdf:tounicode EUC-UCS2}}
\else
\AtBeginShipoutFirst{\special{pdf:tounicode 90ms-RKSJ-UCS2}}
\fi
%ナビゲーションバー非表示
\setbeamertemplate{navigation symbols}{}
%既定をゴシック体に
\renewcommand{\kanjifamilydefault}{\gtdefault}
%タイトル色
\setbeamercolor{title}{fg=structure, bg=}
%フレームタイトル色
\setbeamercolor{frametitle}{fg=structure, bg=}
%スライド番号のみ表示
%\setbeamertemplate{footline}[frame number]
%itemize
\setbeamertemplate{itemize item}{\small\raise0.5pt\hbox{$\bullet$}}
\setbeamertemplate{itemize subitem}{\tiny\raise1.5pt\hbox{$\blacktriangleright$}}
\setbeamertemplate{itemize subsubitem}{\tiny\raise1.5pt\hbox{$\bigstar$}}
% color
\newcommand{\red}[1]{\textcolor{red}{#1}}
\newcommand{\green}[1]{\textcolor{green!40!black}{#1}}
\newcommand{\blue}[1]{\textcolor{blue!80!black}{#1}}

\title[凸解析学における漸近挙動]{凸解析学における漸近挙動}
\subtitle{Introduction of Asymptotic Cones}
\author[岩本 崚汰]{岩本 崚汰}
\institute[新潟大学大学院]{新潟大学大学院自然科学研究科}
\date{March 14, 2023}

\begin{document}
\maketitle

\begin{frame}{目次}
    \tableofcontents
\end{frame}

\section{動機づけ (Motivation) }
\begin{frame}{目次}
    \tableofcontents[currentsection]
\end{frame}

\begin{frame}{動機づけ (Motivation) }

  漸近推 (Asymptotic cones) の定義に入る前に、一般的な点列の収束について考える。

  \begin{block}{定義 1.1}
    ある点列 $\{ x_k \}$ がある点 $x$ に収束するような部分列を持つ時にこの点$x$ を点列 $\{ x_k \}$ の収積点と呼ぶ。
  \end{block}

  \begin{block}{命題 1.2}
    $\mathbb{R} ^n$ の実ベクトル空間において、ある点列 $\{ x_k \}_{k \in \mathbb{N}}$ がある点 $x$ への収束することと、その点列が有界で唯一つの収積点を持つ、ということが同値である。

    収積点
  \end{block}

  一般に、$\mathbb{R} ^n$ の実ベクトル空間である点への収束性を考える場合、その集合の有界性と唯一つの収積点を持つ、ということが必要である。
\end{frame}

\begin{frame}{動機づけ (Motivation) }
  \begin{alertblock}{注意}
    ここで点列が有界であることから、ボルツァーノ・ワイエルシュトラスの定理より、収束する部分列が存在することが言える。
  \end{alertblock}

  では、与えられた点列に有界性がない場合はどうすればいいのか?
\end{frame}

\begin{frame}{動機づけ (Motivation) }
  例: $D = \{(x,y) \:|\: y=x^2\}$

\end{frame}


\section{漸近錐とは (What's Asymptotic cones?) }
\begin{frame}{目次}
    \tableofcontents[currentsection]
\end{frame}

\begin{frame}{漸近錐とは (What's Asymptotic cones?) }
    \begin{itemize}
    \item アイテム1
    \item \alert{アイテム2}
        \begin{itemize}
        \item アイテム1
        \item \alert{アイテム2}
            \begin{itemize}
            \item アイテム1
            \item \alert{アイテム2}
            \end{itemize}
        \end{itemize}
    \end{itemize}
    \[
    \bm{x}^\top\bm{y}
    \]
    \begin{enumerate}
    \item abcde
    \item \structure{ABCDE}
    \item
    \end{enumerate}
\end{frame}

\section{漸近関数とは (What's Asymptotic functions?) }
\begin{frame}{目次}
    \tableofcontents[currentsection]
\end{frame}

\begin{frame}{漸近関数とは (What's Asymptotic functions?) }
    \begin{itemize}
    \item アイテム1
    \item \alert{アイテム2}
        \begin{itemize}
        \item アイテム1
        \item \alert{アイテム2}
            \begin{itemize}
            \item アイテム1
            \item \alert{アイテム2}
            \end{itemize}
        \end{itemize}
    \end{itemize}
    \[
    \bm{x}^\top\bm{y}
    \]
    \begin{enumerate}
    \item abcde
    \item \structure{ABCDE}
    \item
    \end{enumerate}
\end{frame}

\section{今後の目標 (Next goal) }
\begin{frame}{目次}
    \tableofcontents[currentsection]
\end{frame}

\begin{frame}{今後の目標 (Next goals) }
    \begin{itemize}
    \item アイテム1
    \item \alert{アイテム2}
        \begin{itemize}
        \item アイテム1
        \item \alert{アイテム2}
            \begin{itemize}
            \item アイテム1
            \item \alert{アイテム2}
            \end{itemize}
        \end{itemize}
    \end{itemize}
    \[
    \bm{x}^\top\bm{y}
    \]
    \begin{enumerate}
    \item abcde
    \item \structure{ABCDE}
    \item
    \end{enumerate}
\end{frame}

\end{document}
